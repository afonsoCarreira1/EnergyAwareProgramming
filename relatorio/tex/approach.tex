\usepackage{xcolor}

\usepackage[most]{tcolorbox}
\usepackage{minted}
\usepackage{placeins}
\usepackage{float}
\floatstyle{plain}
\newfloat{listing}{H}{lop}
\floatname{listing}{Listing}
\tcbuselibrary{listings, breakable}

\chapter{Approach}\label{chapter:approach}

\begin{figure}[htbp]
  \centering
  \includegraphics[width = .8 \textwidth]{figures/main_modules.pdf}
  \caption{Main modules}
  \label{fig:main_modules}
\end{figure}
As described in the previous sections, the aim of this work is to make developers aware of the energy consumption of their programs. By using a simple and practical tool, they can quickly and accurately estimate their program's energy consumption. This allows them to get immediate feedback on energy consumption with every code change, facilitating energy-efficient development. It is important to note that this tool serves as a guide, providing energy consumption estimates to raise awareness rather than dictate action. Ultimately, it is up to developers to decide whether to prioritize performance, energy efficiency or any other factor. For example, if a program only needs to run within a certain timeframe and can afford a slight reduction in performance, developers may choose to trade some performance for improved energy efficiency, making more informed decisions thanks to the insights provided by the tool.

To provide this insight to developers it is necessary to build a tool that can provide all of that. The tool needs to be practical, which means that integrating it in an IDE is recommended. With this the developer only needs to download an extension for an IDE and will access to the insights provided by the tool.


The tool will be a VSCode extension built using the Language Server Protocol (LSP). While VSCode may not be the most commonly used IDE for Java projects compared to Eclipse or IntelliJ IDEA, it provides a much simpler and more accessible environment for developing and testing extensions. This will make it accessible to most developers wanting feedback on the energy consumption. To make it fast, it will use static analysis to parse the code into an AST, from there it is capable of analyzing the code and using an inference function it will output the estimated cost.
Although the tool is initially built for VSCode, its use of the LSP standard means it can be extended to other IDEs like Eclipse or IntelliJ IDEA. Since LSP handles the communication between the language server and the development environment, porting the tool to new IDEs would primarily involve adapting the user interface and integration specifics, rather than reworking the core logic.


Many devices rely on Java and the JVM, so it is important that the code they run is energy efficient. Several factors can affect the power consumption of Java applications, including the behavior of the garbage collector and the efficiency of the memory management system ~\cite{10.5555/1267847.1267870} making it difficult to predict the power consumption of Java programs. This unpredictability highlights the need for a specialized tool to accurately measure and analyze power consumption so that developers can optimize their applications for energy efficiency.
Java is an excellent choice for developing this tool because of its high interoperability with various operating systems and its widespread usage across the globe, making it a reliable and option. It has a wide range of useful libraries (JRAPL, JoularJx, Jalen) that help to measure energy accurately, and Java's typing and object-oriented features make the code easier to maintain and extend, so the tool can evolve with new energy metering standards and technologies. 

There are several Java parsing tools available, such as WALA~\cite{wala_main}, SootUp~\cite{sootup_main}, Spoon~\cite{spoon_main} and JavaParser~\cite{javaParser}. WALA and SootUp are primarily designed for analyzing Java Bytecode and are generally more complex to use. For this project, Spoon was chosen because it is a user-friendly tool that facilitates easy retrieval and manipulation of the AST from Java source code. Both JavaParser and Spoon support AST manipulation and code generation. However, Spoon provides a deeper, type-aware metamodel and built-in templating features. These features make Spoon more suitable for generating code that conforms to Java’s syntactic and semantic rules, especially in complex transformation scenarios.

In order to build the extension, it was necessary to build a system architecture capable of providing as final output the functioning tool. The architecture follows different stages, that were deeply analyzed before moving to the next one.
The architecture is divided in 4 main modules, the Program Generator, Orchestrator, Parser and Tool.



\section{Stage 1: Program Generator} \label{sec:work_stage1_program_generator}




To use machine learning models to predict energy consumption, it is necessary to have a large amount of data. This data can be obtained by generating programs that are then executed to collect energy profiles. The program generator is responsible for creating these programs, which are then used to train the machine learning models. The generator is designed to be flexible and adaptable, allowing it to generate a wide variety of programs based on different templates and input parameters. The architecture of the Program Generator can be seen in Figure~\ref{fig:program_generator}.


The program generator works alongside with Java Spoon to make it the more general as possible allowing custom programs to be mass generated.

The generator is capable of generating programs for custom, developer-created, Java classes, as well as for the most common APIs, such as collections interface implementations (Lists, Sets, Maps), and utility classes like Math, Base64, Duration. It can also be configured to analyze all public methods of a class or just specific ones.

\begin{figure}[htbp]%[H]
    \centering
    \includegraphics[width = 1 \textwidth]{figures/program_generator.pdf}
    \caption{Program Generator}
    \label{fig:program_generator}
\end{figure}


\subsection{Template Creation} \label{sec:work_stage1_template_creation}

%\FloatBarrier

%\begin{listing}[htbp]
%\begin{minted}[linenos, fontsize=\small, frame=none, bgcolor=white,breaklines=true,breakanywhere=true]{java}
%
%public class ArrayList_add_java_lang_Object_ {
%    public static void main(String[] args) throws Exception {
%        int iter = 0;
%        try {
%            BenchmarkArgs[] arr = new BenchmarkArgs["numberOfFunCalls"];
%            populateArray(arr);
%            TemplatesAux.sendStartSignalToOrchestrator(args[0]);
%            TemplatesAux.launchTimerThread(1100);
%            iter = computation(arr, arr.length);
%        } catch (OutOfMemoryError e) {
%            TemplatesAux.writeErrorInFile("ArrayList_add_java_lang_Object_", "Out of memory error caught by the program:\n" + e.getMessage());
%        } catch (Exception e) {
%            TemplatesAux.writeErrorInFile("ArrayList_add_java_lang_Object_", "Error caught by the program:\n" + e.getMessage());
%        } finally {
%            TemplatesAux.sendStopSignalToOrchestrator(args[0], iter);
%        }
%    }
%
%    static class BenchmarkArgs {
%        public ArrayList<changetypehere> var0;
%
%        public changetypehere var1;
%
%        BenchmarkArgs() {
%            this.var0 = new ArrayList();
%            CollectionAux.insertRandomNumbers(var0, "ChangeValueHere1_changetypehere", "changetypehere");
%            this.var1 = "ChangeValueHere2_changetypehere";
%        }
%    }
%
%    private static void arrayList_add_java_lang_Object_(ArrayList var, changetypehere arg0) {
%        var.add(arg0);
%    }
%
%    private static int computation(BenchmarkArgs[] args, int iter) {
%        int i = 0;
%        while (!TemplatesAux.stop && i < iter) {
%              arrayList_add_java_lang_Object_(args[i].var0, args[i].var1);
%               i++;
%        }
%        return iter;
%    }
%
%    private static void populateArray(BenchmarkArgs[] arr) {
%        for (int i = 0;i < "numberOfFunCalls";i++) {
%          arr[i] = new BenchmarkArgs();
%        };
%    }
%
%    private String input1 = "ChangeValueHere1";
%    private String input2 = "ChangeValueHere2";
%}
%
%
%\end{minted}
%\caption{Example of template for List.add(Object)}            
%\label{lst:template_example}
%\end{listing}

The first step to generate multiple programs is to first create an intermediate template capable of holding the necessary code that will later be used for energy profiling.

\wo{O parágrafo abaixo está um bocado confuso. acho melhor reescrever.}

To generate programs for a specific class, the user has to input the name of the collection or the name of the custom program it wants to generate. For the collections of Lists, Sets and Maps, the generator is prepared to create more than one implementation of those collections, for example, if the objective is to generate programs to the List collection, the program generator will use the \texttt{ArrayList}, \texttt{LinkedList}, \texttt{Vector} and \texttt{CopyOnWriteArrayList}. The user can also input the name of the method to be analyzed, or simply gather all the available methods in the class, which are then found using Spoon.
After the search for the methods, the generator has access to all the methods it will analyze and their input parameters. Since it has access to the whole class, it can see how its constructors are called, and use them if any of the methods parameters requires. It recursively calls constructors if needed, making it very versatile to use. After identifying the methods that will be targeted, it starts by creating templates for each of them. The templates are Java classes that contain the necessary code to run the method, and placeholders for the inputs and types. The templates are structured so that they can be easily modified later, allowing the generator to create multiple programs based on the same template. The templates are designed following the structure below:

\begin{itemize}

\item Input placeholders: Placeholders that will be later changed with real values for inputs, in this case inputs are variable values. 

\item Type placeholder: Placeholders that will later be replaced with Java wrapper classes.
  
\item Array Size placeholder: Placeholder that will later be replaced with the value of an array size. (more details in~\ref{sec:work_stage2_orchestrator}). 

\end{itemize}

Also, the template is structured so that the programs will work in harmony with the \textbf{Orchestrator}, that will extract the energy profile for each program (see Section~\ref{sec:work_stage2_orchestrator}), so when the placeholders are replaced with actual values, the orchestrator can run the program, and communicate with them.


\begin{listing}[H]
\noindent\rule{\linewidth}{0.4pt}
\begin{minted}[linenos, fontsize=\small, frame=none, bgcolor=white,breaklines=true,breakanywhere=true]{java}
static class BenchmarkArgs {
        public ArrayList<changetypehere> var0;

        public changetypehere var1;

        BenchmarkArgs() {
            this.var0 = new ArrayList();
            CollectionAux.insertRandomNumbers(var0, "ChangeValueHere1_changetypehere", "changetypehere");
            this.var1 = "ChangeValueHere2_changetypehere";
        }
    }
\end{minted}
\noindent\rule{\linewidth}{0.4pt}
\caption{Example of variable placeholders creations}            
\label{lst:var_placeholders}
\end{listing}

Templates will have the definitions of a class that holds the necessary variables that the method under evaluation needs to run. Listing~\ref{lst:var_placeholders} shows an example of template to evaluate the method \texttt{List.add(Object)}. It follows the algorithm described in Algorithm \ref{alg:test_generation}. First it creates the list with the smallest constructor possible, then if the variable is a collection (List, Set, Map), or an array, it calls a custom-made method that populates collections with random values of a given type, and then it starts creating variables of parameters that the method \texttt{List.add(Object)} uses. The placeholder \texttt{ChangeValueHere1} will change to a random number, it contains a number \texttt{1} because it represents the input number of the method that will later help the model training understand how inputs can affect energy consumption. The placeholder \texttt{changetypehere} later changes to a type. The template after the transformation can be seen in the Listing~\ref{lst:var_placeholders_replaced}



\begin{algorithm}[H]
\caption{Template Fullfillment Algorithm}
\label{alg:test_generation}
\begin{algorithmic}[1]
    \Statex \textbf{Given:}
    \Statex \hspace{\algorithmicindent} $collection \coloneqq \{\text{List}\}$, collection selected.
    \Statex \hspace{\algorithmicindent} $ds \coloneqq \{\text{List, Map, Set, arrays}\}$, existent data structures.
    \Statex \hspace{\algorithmicindent} $methods \coloneqq \{\text{add, size, get}, \dots\}$, the set of methods of the classes.

\Procedure{FullfillTemplate}{}
    \ForAll{method \textbf{in} methods}
        \ForAll{parameter \textbf{in} \texttt{method.parameters}}
            \State textit{var} $\gets$ \Call{ParameterCreation(\texttt{parameter.type})}
            \If{$parameter.type \in ds$}
                \State \Call{FillWithRandomValues}{var, maxSizePlaceHolder, typePlaceHolder}
            \EndIf
        \EndFor
    \EndFor
\EndProcedure

\vspace{1em} 

\Procedure{ParameterCreation}{\textit{type}}
    \If{\Call{IsPrimitive}{type}}
        \State \Return type
    \Else
        \State dependencies $\gets$ \Call{GetConstructorArguments}{type}
        \State args $\gets$ empty list
        \ForAll{dep \textbf{in} dependencies}
            \State arg $\gets$ \Call{ParameterCreation}{dep}
            \State Append arg to args
        \EndFor
        \State instance $\gets$ \Call{Instantiate}{type, args} \Comment{Create an instance, using the smallest constructor.}
        \State \Return instance
    \EndIf
\EndProcedure

\end{algorithmic}
\end{algorithm}

\begin{listing}[H]
\noindent\rule{\linewidth}{0.4pt}
\begin{minted}[linenos, fontsize=\small, frame=none, bgcolor=white,breaklines=true,breakanywhere=true]{java}
static class BenchmarkArgs {
        public ArrayList<Integer> var0;

        public Integer var1;

        BenchmarkArgs() {
            this.var0 = new ArrayList();
            CollectionAux.insertRandomNumbers(var0, 1000, "Integer");
            this.var1 = 10;
        }
    }
\end{minted}
\noindent\rule{\linewidth}{0.4pt}
\caption{Example of variable placeholders replaced}            
\label{lst:var_placeholders_replaced}
\end{listing}



It is worth mentioning that the types used by the generator are the Java wrapper classes, which are object representations of the primitive types. Using these types it is possible to achieve a more general generator, as every program can use them and if other custom types were used it would make the generator more complex and not general.

What mostly differs from template to template is the number of variables used, because different methods have different parameters, so the template can have more or less input placeholders, also the type placeholder changes according to the methods types and parameters.


Creating the template for each method allows cutting off time of the program generation by only having to replace values of the placeholders instead of needing to create the whole program all over again, since the programs for the same method only differ in inputs, types and array size, maintaining all the structure.

\subsection{Input Tester} \label{sec:work_stage1_input_tester}

An important aspect of the program generator are the inputs it generates. Every method has its on funcionality that may be dependent on the input values.  To be able to better generalize, it is important to find the method input values upper limit. Knowing the maximum size that different parameters can have is fundamental as it needs to be representative, so the energy profiles can cover more cases, but not to large so that the programs start to get out of memory or taking too much time to complete.
The limit definition works by using binary search. It has an inital threshold on the inputs (e.g., 1-100,000 to numerical values) and it starts by trying to run the program with the half of the maximum upper limit threshold (e.g., 50,000). 

It can be important, for pratical reasons, to impose a threashold on the execution time (in this project, we established the threshold as 10 seconds based on empirical experimentation). If the program runs successfully, it will increase the input by half, and if it fails, it will decrease the input by half. This process continues until the maximum value for the input is found.

If the method that is being tested has more than one input, the input tester is responsible to find the upper limit for each parameter individually. First it sets all the input values to 1 and then starts the binary search individually for each of the parameters one by one while leaving the other parameters with the value of 1. Although this is a simplification, this method avoids having to find multiple combinations of parameters which would increase the time complexity exponentially. More robust solutions (such as using a combinatory search) could be implemented in the future. Nevertheless, even using the simplified solution, the process of finding the max inputs is time-consuming. To improve its performance, when the maximum value is found, the values of the input type, maximum value and order (e.g., first, second or third parameter), are stored in a file. This makes subsequent executions much faster. It is worthy to mention that the maximum inputs found depend heavily on the machine where the program generation is taking place, as different hardware will change the maximum values allowed for the inputs. Example 1 illustrates how the input tester works for the method \texttt{List.add(index, Element)}.

\begin{tcolorbox}[
    title=Example 1: Input Testing Process for \texttt{List.add(index, Element)},
    colback=gray!5!white,
    colframe=gray!75!black,
    fonttitle=\bfseries,
    breakable,
    label={box:add-method-testing}
]

Consider analyzing the \texttt{add} method of a list with the following parameters:
\begin{itemize}
    \item \texttt{arg0}: Size of the list (integer)
    \item \texttt{arg1}: Index at which to insert the new value (integer)
    \item \texttt{arg2}: Value to be added (integer)
\end{itemize}

\textbf{Step 1 – Varying \texttt{arg0} (list size):}

\begin{itemize}
    \item Iteration 1: \texttt{arg0 = 25,000}, \texttt{arg1 = 1}, \texttt{arg2 = 1}
    \item Iteration 2: \texttt{arg0 = 12,500}, \texttt{arg1 = 1}, \texttt{arg2 = 1}
    \item Iteration 3: \texttt{arg0 = 6,250}, \texttt{arg1 = 1}, \texttt{arg2 = 1}
    \item $\vdots$
    \item Final Iteration: \texttt{arg0 = 1,700}, \texttt{arg1 = 1}, \texttt{arg2 = 1}
\end{itemize}

\textbf{Step 2 – Varying \texttt{arg1} (index):}

\begin{itemize}
    \item Iteration 1: \texttt{arg0 = 1}, \texttt{arg1 = 25,000}, \texttt{arg2 = 1}
    \item Iteration 2: \texttt{arg0 = 1}, \texttt{arg1 = 12,500}, \texttt{arg2 = 1}
    \item $\vdots$
    \item Final Iteration: \texttt{arg0 = 1}, \texttt{arg1 = 1}, \texttt{arg2 = 1}
\end{itemize}

{Note: Since the list size (\texttt{arg0}) remains 1, the maximum valid index (\texttt{arg1}) is constrained to 1. This reveals a limitation in the input testing approach.}

\textbf{Step 3 – Varying \texttt{arg2} (value to be added):}

\begin{itemize}
    \item Iteration 1: \texttt{arg0 = 1}, \texttt{arg1 = 1}, \texttt{arg2 = 25,000}
    \item Iteration 2: \texttt{arg0 = 1}, \texttt{arg1 = 1}, \texttt{arg2 = 37,500}
    \item $\vdots$
    \item Final Iteration: \texttt{arg0 = 1}, \texttt{arg1 = 1}, \texttt{arg2 = 100,000}
\end{itemize}

\textbf{Final stored input limits:}
\begin{itemize}
    \item \texttt{arg0} — \texttt{integer} — \texttt{1,700}
    \item \texttt{arg1} — \texttt{integer} — \texttt{1}
    \item \texttt{arg2} — \texttt{integer} — \texttt{100,000}
\end{itemize}

\end{tcolorbox}


The fact that the input has a pre-determined range it allows using binary search which makes the search much faster than linear search.
Nevertheless, the input search does not come without some limitations. First it is constrained to identifying valid input values within the range. As an example, throughout this project we dealt frequently with lists and collections, which require a minimum size of 1 to function correctly. Although this value can be changed in the future, for now it ensures compatibility with most common data structures. The higher bound was chosen due to empirical evidences, as larger input values would lead to limitating execution times and memory crashes. Another constraint is on how the input handles multiple parameter methods. During its search for a valid input, it needs to fix all the other parameters that is not searching, with a default value (typically the lower bound). This approach simplifies the testing process and improves efficiency by avoiding the exponential complexity of testing all possible parameter combinations. However, it will introduce limitations in cases where the parameters are interdependent, which can lead to not estimating the real highest input possible. 
Despite these limitations, the input search plays a crucial role in ensuring that the program generator produces viable test cases. By identifying input ranges that are both valid and computationally achievable, it reduces the unusable generated programs (e.g., due to timeouts or crashes), and maximizing the number of generated programs, that can be used for energy profiling.



\subsection{Program Generation} \label{sec:work_stage1_program_generation}

Finally, when the templates are created, and the maximum inputs are found the program generation can finally begin. This part consists on picking every template created and replacing the placeholder values with actual values created by a random number generator.

It starts by looping through the types and changing them to the Java wrapper classes, then it loops through the array size. Lastly, it loops through the input sizes determined by the input tester and can repeat this process a configurable number of times. By increasing the number of iterations, results in more programs being generated with random inputs, constrained by the previously identified input bounds. \wo{desenha um algoritmo como eu fiz anteriormente para mostrar como isso funciona.}

This process easily generate thousands of programs which are crucial to train machine learn models that are able to predict energy consumption. When generating programs, a number is chosen to balance the requirements for effective model training while minimizing the time spent on input testing and collecting energy profiles. Afterwards, that the programs are ready to be compiled and used.

\usepackage{array}
\usepackage{booktabs}
\usepackage{xcolor}
\usepackage{colortbl}

\section{Stage 2: Orchestrator} \label{sec:work_stage2_orchestrator}

\begin{figure}[htbp]
  \centering
  \includegraphics[width = 1 \textwidth]{figures/orchestrators_process.pdf}
  \caption{Orchestrator Workflow}
  \label{fig:orchestrators_process}
\end{figure}


Having all the programs generated, it is possible to move to the next stage. Gathering the energy profiles for each of the generated programs. This step is necessary, of course, because to give energy consumption estimates, first it is necessary to obtain energy profiles~\cite{10.1145/2884781.2884869,8816747}, so the machine learning models can obtain the most accurate results possible. To make this task as easily as possible, a process was implemented, so the task was made automatically, taking into account what was said in \ref{sec:background_benchmarking}. 

Note: Although this part was developed chronologically before Section \ref{sec:work_stage1_program_generator}, it is presented as Stage 2 to reflect the logical structure of the system: programs must first be generated before they can be energy profiled. For this reason, the program generator templates are designed to integrate directly with the orchestrator.

As described in \ref{sec:background_energy} there are several tools capable of performing dynamic energy measurements. In our case we choose PowerJoular, it has good features that caught our attention, for example being a command line tool that could be easily integrated in almost every programming language, it stores the energy used in a CSV file (easy to work with), capable of only reading energy of running processes and so on, as explained before. So, PowerJoular is the tool that will measure the energy of our programs.

Since PowerJoular is a command line tool, it is launched as a process, and then it can be killed whenever needed because it is a process as well. This allows to measure not only programs/processes energies but have a more precise measurement, as it is possible to call PowerJoular to measure a specific computation and then kill it when the computation is finished.

With all of this in mind a process was built. The process uses an orchestrator that is responsible for invoking the target program and the measurement tool (PowerJoular) to accurately measure the energy consumption of the program or the specific computation being analyzed within it.

In the program generator it was shown an example of a template \ref{lst:template_example}. This structure was carefully crafted in order to work with the orchestrator.



When measuring energy, there is a particular problem with computation. It is too fast. It is really difficult to measure a single operation of, for example, a List.add(Object) with most tools, as it simply to fast for the tool to capture, and if the tool was able to capture it, the energy measured would have too much noise to be considered. So, it surges the need to loop through the method until the tool (PowerJoular) is capable of getting its energy and then dividing the total energy by the number of times it looped. This can work, but it brings other errors that will need to be considered.

If we approach the measurements with the loop technique, first we need to create the variables needed for the method target to analysis, like shown in \ref{lst:var_placeholders}. But there is a problem, the methods are treated like a black box, it is not possible to know what the methods are going to do with the parameters or with its variables, for example, the method List.size() it receives nothing, and return the size of the list which is fine. However, this raises problems with other methods, such as List.add(Object), maybe not in the first iterations. But what if the loop iterates too many times? It will make the list much bigger than the initial list, which again will cause differences in the energy measurements and later in the energy predictions. 
Now, suppose we have a custom method called sort that takes a randomly ordered list and sorts it. This introduces a problem when measuring its energy consumption. On the first run, the method will sort the unsorted list, which may take significant time depending on the randomness of the data and the sorting algorithm used. However, on subsequent runs with the same list, the input is already sorted, meaning the sort method will likely complete much faster or with minimal effort. This difference can result in misleading or inconsistent measurements.

\begin{figure}[htbp]
  \centering
  \includegraphics[width = 1 \textwidth]{figures/array.pdf}
  \caption{Array example}
  \label{fig:array}
\end{figure}

To prevent this from happening the approach used was to create an array that holds the parameters \ref{fig:array}. Basically, the parameters are created into an array, all with the same values, but with different references. Now, if one of the elements is changed during its execution on the method, it will not affect the other's execution. Of course, since now there are multiple references of the same values, the memory usage is much higher, so that is why there is an array size, also empirically chosen, to avoid memory errors and avoid having the array so small the PowerJoular would not have time to perform the measurements. There are, for now, 3 different array sizes (75_000, 100_000, 150_000). PowerJoular by default needs to run for 1 second to create the CSV file that has the energy used for the target process, so these sizes aim so that looping the array takes at least a second. In some cases, the CSV file may not be generated because the loop executes too quickly. In such instances, the energy reading will be assumed to be 0 J, as the execution was too fast to record a value. Using an array in this case is preferable to a list, as it introduces less overhead and has a smaller impact on energy measurements.


\begin{listing}[H]
\begin{minted}[linenos, fontsize=\small, frame=none, bgcolor=white,breaklines=true,breakanywhere=true]{java}
private static int computation(BenchmarkArgs[] args, int iter) {
        int i = 0;
        while (!TemplatesAux.stop && i < iter) {
              arrayList_add_java_lang_Object_(args[i].var0, args[i].var1);
               i++;
        }
        return iter;
    }
\end{minted}
\caption{Computation method}            
\label{lst:Computation_method}
\end{listing}

The computation method shown in Listing \ref{lst:Computation_method} illustrates how each profiling method operates independently of the specific method being evaluated. It attempts to execute the target function repeatedly for approximately one second. In this particular case, the target function performs the operation \texttt{var.add(arg0);} and returns the number of iterations completed, which is then used for further calculations.


With the process structure now defined, we can proceed to explain how it functions.

The workflow of this step can be described as follows:

\begin{itemize}
  \item The orchestrator launches a command to start the target Java Program and waits a signal.
  \item The Java program starts and setup the necessary elements to run (creating all the variables, reading/writing files, populating the array, etc.) and then before starting the computation it wants to measure, it sends a start signal to the orchestrator to start monitoring, and waits for 100 milliseconds.
  \item The orchestrator receives the start signal and reads the PID, which is stored in a file during the target program setup. Finally, it starts PowerJoular using that PID. Then it waits for the stop signal.
  \item The Java program will run until it finishes the computation. The computation runs for a maximum of one second. Then the number of iterations are stored in a file and the stop signal is sent back to the orchestrator.
  \item The orchestrator on receiving the stop signal, first stops PowerJoular and then stops the target program, if needed. Then it parses the target program to extract its features, combines them with the energy information stored in the files created by PowerJoular, stores it in a CSV file.
\end{itemize}

All these steps are performed for each generated program. At the end of the process, a log file is created containing key information, including all the programs used, the PowerJoular files generated, temporary files, error logs, and features.
Additionally, features from certain programs are merged at the final stage. While each method is initially trained individually, it is not useful to treat LinkedList.add(Object) and ArrayList.add(Object) as entirely separate cases. Both represent the same List.add(Object) method, differing only in their characteristics specific to their implementation. Therefore, once all energy profiles have been collected, a merging step is performed to consolidate these related methods.

The features extracted from the parser are the following:

\begin{table}[htbp]
\centering
\caption{Features extracted}
\label{tab:features_extracted}
\begin{tabular}{||l|c||}
\hline\hline
\textbf{Feature Name} & \textbf{Description} \\
\hline\hline
VariableDeclarations & Number of variable declarations in the code. \\
\hline
Assignments & Number of assignment operations (=) in the code. \\
\hline
BinaryOperators & Number of binary operators used (e.g., +, -, *, /, \&\&). \\
\hline
MethodInvocations & Number of method calls in the code. \\
\hline
CustomMethodsUsed & Number of user-defined (custom) methods used in the code. \\
\hline
CustomObjectsUsed & Number of user-defined (custom) objects used in the code. \\
\hline
BranchCount & Number of branching statements ( if, else if, else ) in the code. \\
\hline
LoopCount & Number of loops ( for, while, do-while ) in the code. \\
\hline
loopMaxDepth & Maximum nesting depth of loops in the code. \\
\hline
cyclomaticComplexity & The cyclomatic complexity of the code (number of independent execution paths). \\
\hline
literalCount & Number of literals (numbers, strings, boolean values) in the code. \\
\hline
VariableCount & Total number of variables in the code. \\
\hline
Reassignments & Number of times variables are reassigned a new value. \\
\hline
variableTypes & Types of variables used (e.g., int, String, List$<$String$>$). \\
\hline
importsUsed & List of imported Java packages/libraries in the code. \\
\hline
java.langMethodUsed & If the method is from java.lang, the specific function name (e.g., ArrayList.get). \\
\hline\hline
\end{tabular}
\end{table}

\section{Stage 3: Model Training} \label{sec:work_stage3_model_training}

Now, that all the energy profiles are collected it is possible to finally start training the models. As explained in \ref{sec:background_machine_learning} there are a lot of possible ways of using machine learning, however in this case the approach that best fit our need is supervised machine learning. So, some models were trained to see how good they performed using the data collected.

First there is a merge of features. While each method is initially trained individually, it is not useful to treat LinkedList.add(Object) and ArrayList.add(Object) as entirely separate cases. Both represent the same List.add(Object) method, differing only in their characteristics specific to their implementation. Therefore, once all energy profiles have been collected, a merging step is performed to consolidate these related methods.

All these features are stored in a new CSV and for the python program in a Data frame, to be processed.

This part was developed in python using some libraries specialized on machine learning, like scikit-learn (sklearn)~\cite{scikit-learn} which contains some models and functions to evaluate the models, and PySR~\cite{cranmer2023interpretablemachinelearningscience} which was already explained in \ref{sec:background_machine_learning} an is also already implemented.

In this phase the models tested were: Decision Tree Regressor, Random Forest, Gradient Boosting, Linear Regression, and PySR.

First the values of the MSE and R2 are evaluated by the default values of sklearn, then in a second pass, GridSearch was used, which is a function of sklearn that tries to find the best parameters for a model. After that the scores and models are saved. The GridSearch does not work with PySR as they are from different libraries so, the parameters for PySR were manually set.

In the end the chosen model was PySR as it can represent the predictions in expressions which can help the users to try to understand why the code is using more energy. It has a nice feature of allowing to balance complexity and accuracy. And can easily be used in another code language as it is represented as a mathematical expression.





%\section{Step 3: Implementation and testing} \label{sec:work_step3_implementation_and_testing}
%
%Once the main components of the tool are built, they need to be assembled into the extension. When using it, the developers should be able to see the total energy estimate of their code in the IDE, and it should also show the estimates for each function and its most energy consuming lines.
%The estimate alone may be enough to understand if the code is high or low in energy consumption, for example, if the developer has two implementations of the same function and they both give different values, it may be easy to understand which one consumes the most. However, this may not always be the case, so the tool will also provide some information to help the developer know how good or bad the energy efficiency of the code is.
%
%Another important step is to test and ensure that the tool performs as expected on most machines, delivers the most accurate estimations possible, and undergoes a final comparison with other tools to evaluate its effectiveness.
%