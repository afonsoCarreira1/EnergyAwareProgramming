% Conclusao

\chapter{Conclusion}\label{chapter:conclusion}

The techniques and tools in past and recent research have been studied, and this work improves on those different research and tools, by providing an easy-to-use tool that is also easy accessible and provides energy estimations.
For now, the tool to perform the energy profiling as been selected, (PowerJoular) and a process to create the profiles has been made. This process, as explained in section \ref{sec:work_step1_energy_profiling}, allows targeting specific parts of programs, or the entire program, measure the energy consumption and extract important features. The selection of the tools had some setbacks. When testing PowerJoular it was noticed that it didn't work correctly when using python as an orchestrator, obtaining different measurements than the other orchestrators, (Java, C, Bash), as explained in section \ref{chapter:results}. Other than that, the tool worked fine, and it's capable of completing its task of energy profiling. The next step is to start the development of the energy inference function, by collecting the required data. When completed, it is expected that the tool can raise energy awareness among developers, and help facilitate the process of making energy efficient code.
