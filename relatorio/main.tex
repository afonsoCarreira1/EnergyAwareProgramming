%
% Modelo para relatório de trabahlo final das disciplinas de Projecto dos
% mestrados do DI:
% - Engenharia Informática
% - Informática
% - Segurança Informática
% - Ciência de Dados
%
% Incorpora elementos impostos pelo Regulamento de Estudos Pos-Graduados da
% Universidade de Lisboa
%(Deliberacao 1506/2006 - Diario da República, 2.a série - n.o 209 -
% 30 de Outubro de 2006)
%
% Versão Setembro 2024 
%
% Algumas  dicas:
%  - Na pasta ./tex deverão estar os vários ficheiros latex que constituem o relatório
%  - Primeiro os ficheiros:
%		- capa.tex
%		- agradecimentos.tex
%		- resumo_pt.tex
%		- resumo_ing.tex
%		- No início de cada um destes ficheiros é mencionado o que é necessário/possível alterar
%  - Depois vêm os ficheiros com os capítulos, organizados da forma que entenderem e incluídos de forma apropriada no local indicado neste ficheiro main.
%  - Na pasta ./pic deverão ser colocadas as figuras
%  - A bibliografia deverá estar num ficheirona mesma directoria que este main.tex	
%
%%%%%%%%%%%%%%%%%%%%%%%%%%%%%%%%%%%%%%%%%%%%%%%%%%%%%%%%%%%%%%%%%%%%%%
\documentclass[11pt,openright,twoside]{report}

\usepackage[utf8]{inputenc}
% Quem tiver problemas com os acentos, trocar utf8 por latin1

\usepackage[portuguese,english]{babel}
\usepackage{times}

% Isto é só para gerar o texto dummy do exemplo. Remover da tese.
%\usepackage{blindtext}

\usepackage{graphicx}

% Indice remissivo
\usepackage{makeidx}
\makeindex

% Glossario
\usepackage{glossaries}
\makeglossaries

% Links
\usepackage{hyperref}

% Package para cabecalhos
\usepackage{fancyhdr}
\usepackage{lastpage}

\fancyhf{} %
\lhead{\nouppercase {\leftmark}} %
\rhead{\nouppercase {\bf \thepage}}
\renewcommand{\headrulewidth}{0.1pt}

% Comando para inserir pagina em branco (inserida na numeracao, mas sem
% numero impresso) para quando e' preciso obrigar um capitulo a comecar
% do lado direito (pagina impar)
\newcommand{\LIMPA}{
\newpage
\mbox{}
\thispagestyle{empty}
}

% Igual, mas insere pagina com numero impresso (normalmente nao se usa)
\newcommand{\LIMPAC}{
\newpage
\mbox{}
\thispagestyle{plain}
}

%
% ALTERAR AQUI AS INFORMACOES RELATIVAS AO PROJECTO
%
\newcommand{\TITULO}{Título do Trabalho}
\newcommand{\Autor}{Nome Completo do Aluno}

%Orientador e CoOrientador *sem* titulos (e.g. Prof. Doutor)
\newcommand{\Orientador}{Nome Completo do Orientador}
\newcommand{\PCoOrientador}{Nome Completo do Orientador} %se nao se aplicar, nao importa o que aqui esteja

\newcommand{\PEIAnoLectivo}{2023/2024}
\newcommand{\PEIAno}{\Large{2024}}

% Comentar/descomentar conforme conveniente
%\newcommand{\PEITIPO}{DISSERTAÇÃO}
%\newcommand{\PEITIPO}{TRABALHO DE PROJETO}

% Comentar/descomentar conforme conveniente
\newcommand{\PEIIdiomaTese}{\selectlanguage{portuguese}}
%\newcommand{\PEIIdiomaTese}{\selectlanguage{english}}

% Comentar/descomentar conforme conveniente
\newcommand{\MDesignacao}{Designação do Mestrado}
%\newcommand{\MDesignacao}{Engenharia Informática}
%\newcommand{\MDesignacao}{Informática}
%\newcommand{\MDesignacao}{Segurança Informática}
%\newcommand{\MDesignacao}{Ciências de Dados}


% Comentar/descomentar conforme conveniente
\newcommand{\MEspecializacao}{Designação da Especialização / Perfil, se aplicável}
%\newcommand{\MEspecializacao}{Sistemas de Informação}
%\newcommand{\MEspecializacao}{Interacção e Conhecimento}
%\newcommand{\MEspecializacao}{Engenharia de Software }

%\usepackage{ifpdf}
%\ifpdf
%\pdfinfo {
%	/Author (\Autor)
%	/Title (\TITULO)
%	/Subject (\MEspecializacao)
%	/Keywords ()
%	/CreationDate (D:20230825)
%}
%\fi

\usepackage[dvips]{geometry}
\geometry{a4paper=true,portrait=true,left=3cm,right=3cm,top=2.5cm,bottom=3.5cm}

%\title{\PEITITULO}
%\author{\PEIAutor}
%\date{\today}

%% Para o ano da tese ficar mais abaixo, de acordo com o template da FCUL
\fancypagestyle{estilocapa}{\fancyhf{}\renewcommand{\headrulewidth}{0pt}\fancyfoot[CE,CO]{\large{\fontfamily{phv}\fontseries{c}\selectfont \PEIAno}}}



\begin{document}

%Capa e pagina de rosto
%
% capa.tex
%
% Alterar apenas nas zonas indicadas com >>>
% - Existência, ou não, de especialização no mestrado em causa
% - Existência, ou não, de coorientador
%
\selectlanguage{portuguese}

%\pagestyle{empty}
\thispagestyle{estilocapa}
% ----------------------------------------------------------------------
% Capa
\begin{center}
%\fontfamily{lmss}\selectfont
\fontfamily{phv}\fontseries{c}\selectfont
\vspace{1cm}\normalfont\normalfont
\vfill
{\fontfamily{phv}\fontseries{c}\selectfont 
\large\uppercase{Universidade de Lisboa}\\
\vspace{0.2cm}
\large\uppercase{Faculdade de Ciências}\\
\vspace{0.2cm}
\large\uppercase{Departamento de Informática}\\
}
\vspace{0.7cm}
\includegraphics[scale=.45]{figures/logo_fcul.png}\\
%\includegraphics[scale=.6]{pic/logo_ul.png}\\

\vspace{3.1cm}
\vfill
\PEIIdiomaTese
{\LARGE \bf \fontfamily{phv}\fontseries{c}\selectfont \TITULO}


\selectlanguage{portuguese}
\vspace{1cm}
%%

\vspace{1.5cm}
\vfill
\Large{ \fontfamily{phv}\fontseries{c}\selectfont \Autor}\\
\vspace{1.8 cm}
\vfill
\large{\bf{\fontfamily{phv}\fontseries{c}\selectfont Mestrado em \MDesignacao }}\\
% >>>
% Comentar a linha seguinte se o mestrado não tem especialização
%\large{\fontfamily{phv}\fontseries{c}\selectfont \MEspecializacao}\\
\vspace{0.8cm}
\vfill
%% Eliminar na versão definitiva
\normalsize{\fontfamily{phv}\fontseries{c}\selectfont Versão Provisória}\\
%%
\vspace{0.8cm}
\vfill
\large{\fontfamily{phv}\fontseries{c}\selectfont Dissertação orientada por:}\\%[Dissertação orientada]/[Trabalho de Projeto orientado]
%\normalsize{Trabalho de projeto orientado por:}\\
\large{\fontfamily{phv}\fontseries{c}\selectfont Prof. Doutor \Orientador} \\
% >>>
% Comentar a linha seguinte se não há coorientador
\large{\fontfamily{phv}\fontseries{c}\selectfont Prof. Doutor \PCoOrientador} \\
\vspace{1.5 cm}
\vfill

%\vspace{1.5cm}
\vfill
%\large{\fontfamily{phv}\fontseries{c}\selectfont \PEIAno}
\end{center}
% Fim da capa
% ----------------------------------------------------------------------
\setcounter{page}{1}
\pagenumbering{roman}
\newpage
\mbox{}
\newpage




% Agradecimentos
%
% agradecimentos.tex
%
% Alterar apenas nas zonas indicadas com >>>
% - Escolha do título em função da língua principal
% - adicionar texto principal

%\pagestyle{plain}

\vspace*{2cm}
\begin{center}
% >>>
% Comentar as duas linhas que não interessam consiante a língua principal usada no relatório
%\selectlanguage{portuguese}
%\Large \bf Agradecimentos
\selectlanguage{english}
\Large \bf Acknowledgments
\end{center}
\vspace*{1cm} \setlength{\baselineskip}{0.6cm}

% >>>
% substituir a linha seguinte pelo texto pretendido

I want to express my gratitude to my family, parents, sister and grandparents for their constant encouragement, assistance and love towards me throughout my academic career. Their confidence in me has at all times been a source of inspiration.

I would also like to thank my supervisors, Professor Alcides and Professor Wellington, for their perceptive feedback, guidance, and patience throughout the writing of this thesis. Their mentoring was integral to this piece of work's success.

\LIMPA

~
\vfill

%\selectlanguage{portuguese}
\selectlanguage{english}
\begin{flushright}\textit{Dedicatória.}\end{flushright}

%\LIMPA


% Pagina do resumo em ingles
% ----------------------------------------------------------------------
% Página do resumo em Inglês:
\selectlanguage{english}
\vspace*{2cm}
\begin{center}
\Large \bf Abstract
\end{center}
\vspace*{1cm} \setlength{\baselineskip}{0.6cm}

  Energy efficiency in software development is on the rise due to the increasing demand for environmental sustainability and the need to reduce technology operating costs. Despite progress, developers often lack the tools and knowledge to effectively optimize the energy consumption of their programs. This project proposes an IDE extension designed to estimate the energy cost of programs and code snippets through static code analysis. The tool works by performing a static analysis of the code and feeding it to a machine learning model that can quickly estimate the energy cost. The feedback is intended to increase developers' energy awareness and encourage more energy-efficient coding practices.

%Resumo até \textbf{300} palavras. 

\vfill

\begin{flushleft}
\textbf{Keywords:}
Green computing, Static code analysis, Energy prediction, Sustainable software, Energy-aware programming
\end{flushleft}

%\LIMPA
\selectlanguage{portuguese}
% Fim da página do resumo em Inglês.
% ----------------------------------------------------------------------


\pagestyle{plain}

\PEIIdiomaTese

% Indice
\tableofcontents

%\LIMPA

%Lista de figuras
\listoffigures

\addcontentsline {toc} {chapter} {\listfigurename}
%\newpage
\thispagestyle{empty}
%\mbox{}
%\newpage

%Lista de tabelas
\listoftables

\addcontentsline {toc} {chapter} {\listtablename}
%\newpage
\thispagestyle{empty} \mbox{}
%\newpage

% ----------------------------------------------------------------------
% Inicio conteudo - Ficheiros com os capítulos

\pagestyle{fancy}
\cleardoublepage

\setcounter{page}{1}
\pagenumbering{arabic}

% Conteudo (incl. Introducao)
\chapter{Introduction}

In recent years, the use and management of energy has become a global issue. The search is on for renewable energies that reduce the ecological impact on our planet. However, these alternatives are neither as cheap nor as consistent as traditional options. There are many areas in which that may reduce their energy footprint, and the IT sector is one of them.

Saving energy in programs is crucial for the operation of certain devices, such as mobile phones or IoT devices, so certain techniques need to be applied in order to reduce the energy of a program. For mobile devices, companies such as Google and Apple have developed tools\cite{google_adaptive_battery,google_battery_saver,apple_clean_energy, android_power_profiler} to help save energy and while running the applications techniques are already used to save the battery when necessary, but for systems that don't use batteries, such as servers, energy is rarely taken into account when developing a program. This lack of concern or awareness on the part of developers, although it appears to have a small impact, turns out to be quite significant; In 2020 around 7\% of global electricity use was due to information and communications technology, with an anticipated rise in line with the growing demand for new technologies\cite{article}. This trend has become even more significant with the increased use of artificial intelligence \cite{patterson2021carbon}, especially large scale models such as ChatGPT, which require significant computing resources to train and run. These energy-intensive processes contribute significantly to global energy consumption and carbon emissions, raising environmental concerns as AI adoption continues to grow. For instance, training the GPT-3 model required 1,287 MWh of energy, equivalent to the annual energy usage of approximately 117 U.S. households, and produced 552 metric tons of CO2—comparable to the emissions from driving 120 cars for a year. With the release of GPT-4 and ongoing development of even more advanced models, these numbers are expected to rise, further amplifying their environmental footprint. The significant energy demands of data centers lead to considerable heat generation, requiring Heating, Ventilation, and Air Conditioning (HVAC) systems to ensure stable operations. Remarkably, HVAC systems consume approximately 33\% of a data center's total energy, with another 18\% dedicated to Computer Room Air Conditioning (CRAC) units. Servers, which are integral to data center functionality, account for 45\% of the energy usage, even without factoring in AI-driven tasks and complex modeling workloads~\cite{balaras2017high}.

Some key reasons to prioritize energy efficiency in software, whether for mobile systems or data center applications, include:

\begin{itemize}
  \item The dependence of mobile devices on batteries. All mobile devices rely on their batteries, so the software they run needs to make the best use of resources to conserve battery power.
  \item Reducing operating costs in data centers. It is crucial to reduce the operating cost of data centers by using energy-efficient programs. This reduction results in economic benefits for companies and contributes positively to environmental sustainability. 
  \item Reducing energy consumption has a positive impact on our environment by saving energy that can be used more efficiently elsewhere. 
\end{itemize}

When developing a program, most of the time developers optimize for the time the program takes to complete, or the memory it uses, and not so often take into account the energy it uses. 
In the cases where developers actually want to improve the energy efficiency of the code, they normally have difficulties and seek help, relying on blogs, websites and YouTube videos, which in most cases lack empirical evidence, leading to perceptions of improvement rather than measured benefits\cite{10.1145/3154384}. This is due to a lack of knowledge and guidelines, because understanding the energy usage of a program and how to make it more efficient is not trivial, as running the same program multiple times will output different values each time and even if the execution time of the program is reduced, is not guaranteed to also reduce energy consumption. Because of its difficulty, there is still a need for tools that can help with this task\cite{10.1145/2597073.2597110}. 

Also, most current tools can measure the energy of programs and applications as they run (e.g., Android Studio Power Profiler~\cite{android_power_profiler}), but this usually requires extra steps that many developers may not have the time or inclination to take, so there is a need for a tool that can help the developer without the need for extra effort\cite{10.1145/3154384}.

This thesis proposes the development of a tool capable of identifying the energy consumption of methods in programs and presenting this information quickly to programmers, enabling them to make informed decisions in software design. The goal is to create an IDE extension that integrates these functionalities, provides immediate feedback on the energy impact of applications, and allows developers to adjust their code to meet efficiency requirements. The tool should be easy to use, requiring minimal knowledge of energy consumption, while helping programmers understand the energy footprint of their software. This increased awareness will enable them to understand the overall impact of their coding choices on energy consumption and efficiency.

To create this tool, it is essential to understand the current state of the art, including the techniques previously used and the tools currently in use. The tool will employ static analysis techniques to identify which instructions are utilized, and through inferences from previously collected data, indicate the estimated energy levels of the program's execution. The inference will be made using energy data collected from low-level library functions. More complex functions are built on the basis of function composition, which means that, based on the estimated consumption of low-level functions, we can generalize our estimates to more complex functions and ultimately to the program as a whole. 

\section{Motivation}



\section{Objectives}

\blindtext[1]

\blindtext[2]

\section{Contributions}

\Blindtext[2]

\section{Document Structure}

This document is organized as follows:

\begin{itemize}
\item Chapter \ref{chapter:background} – introduces key concepts necessary to fully understand the report. It discusses the challenges of predicting and measuring energy consumption in programs, explores various energy tools and machine learning techniques, and provides an overview of static and dynamic analysis, highlighting their relevance to this work.

\item Chapter \ref{chapter:related_work} – contains the initials solutions proposed to the theme of energy aware programming, how they changed during the years, and what the most recent tools do. And comparing with the proposed tool in this work.

\item Chapter \ref{chapter:methodology} - explains in detail the existing problem and what is the solution, and a detailed explanation on how the solution will be built.

\item Chapter \ref{chapter:results} - reports on the experiments made so far, and the obtained results.

\item Chapter \ref{chapter:conclusion} - summarizes the work completed to date.

\item Chapter \ref{chapter:future_work} - outlines future research directions.

\end{itemize}

%background 
\chapter{Background}\label{chapter:background}

This section provides an overview of the essential concepts and foundational knowledge necessary for understanding the methods and approaches used throughout this thesis. A solid understanding of these concepts is necessary to fully comprehend the work presented. {\color{blue}The section begins with a discussion on benchmarking, inlcuding the best practices for performing benchmarks. It then explores energy profiling, which involves benchmarking but focused on energy measurements. It explains the energy tools capable of performing the energy measurements, how they work, what are the advantages and disadvantages. Additionally, it introduces the concepts of static and dynamic analysis explaining the differences and each is applied. The Language Server Protocol concept is also examined as it has relevance in the context of this thesis. Finally, it has a brief explanation on machine learning and how it can be used in our project.}


\section{Benchmarking} \label{sec:background_benchmarking}

Benchmarking is the process of measuring the performance of a program or system by running a series of tests under controlled conditions. This typically involves evaluating factors such as execution time, memory usage, and energy consumption. The purpose of benchmarking is to gain a clear understanding of how the software behaves in different scenarios, to compare different implementations, and to identify potential areas for improvement.

In this project, the primary goal of benchmarking is to accurately measure the energy consumption of Java applications, while taking into account the factors that can affect these measurements. \wo{aqui voltas a falar dos programas mas desta vez na forma de benchmarks. Me parecem mesmo que estamos a lidar com um gerador de benchmarks e não de programas}

When performing benchmarking (i.e. measuring time, energy, memory, etc.) it is important to have in mind some important information that can affect the measurements. First it matters to know how the programming language being used works, and how it can affect the measurements. It is important to understand which noises can be avoided and which cannot.

As an example, benchmarks in Java, which is the primary programming language used in this project, will have its constraints. Performing benchmarks in Java has its adversities, and needs careful approach when needed. Several papers have studied how to perform better benchmarks~\cite{10.1145/1297027.1297033,10.1145/1167515.1167488} details important aspects on how to correctly benchmark Java applications, and how to avoid common pitfalls.
It is essential to understand the non-deterministic problems that may arrive when benchmarking in Java:


\begin{itemize}
    \item \textbf{Just-In-Time (JIT) compilation}: Automatically performs optimizations in during the compilation, which can affect measurements. Example: In the loop presented in the Listing~\ref{lst:for_loop_example}, the optimization done during the compilation, might understand that the loop is unnecessary and just replace the variable sum with \texttt{int sum = 100;}, avoiding the loop. Basically, the JIT compiler may detect that a loop performs predictable work and optimize it away entirely, especially if the results are unused or can be computed ahead of time.
    
    \begin{listing}[H]
    \noindent\rule{\linewidth}{0.4pt}
    \begin{minted}[linenos, fontsize=\small, frame=none, bgcolor=white,breaklines=true,breakanywhere=true]{Java}
        int sum = 0;
        for (int i = 0; i < 100; i++) {
            sum++;
        }
        System.out.println(sum);
    \end{minted}
    \noindent\rule{\linewidth}{0.4pt}
    \caption{Example for loop that can be optimized by the JIT compiler.}            
    \label{lst:for_loop_example}
    \end{listing}

    
    \item \textbf{Garbage Collection (GC)}: Garbage collection can occur unpredictably, which may impact the accuracy of measurements. For time and energy measurements, it often results in increased values, while for memory measurements, the effect depends on whether garbage collection occurs before or after the measurement period.
    
    \item \textbf{Thread scheduling}: It is also unpredictable when threads stop might have different schedules, resulting in different interactions.
    
    \item \textbf{Java Virtual Machine (JVM)}: Different JVM implementations or configurations (e.g., heap size, GC algorithms, JIT strategies) can result in considerable performance differences. Even the same JVM may behave differently across runs due to internal optimizations and adaptive behaviors. Additionally, JVM startup time can be a problem in performance measurements, especially in short processes, as it introduces overhead not representative of steady-state behavior.

\end{itemize}

Some of these properties cannot be avoided, however there are some procedures that can be taken into consideration to avoid error in the measurements.
To avoid JIT optimizations, harder and complex examples must be used in order to avoid optimizations, for example instead of filling a list in a for loop using the index values, the values added can be random, which will avoid optimizations, as the compiler cannot predict the next number in the iteration. This proves to be effective in energy measurement, because if it is needed to measure the energy consumption of the add method of a list collection, now the process can actually be measured avoiding optimizations, if it was optimized it would be too fast and would not be valid for energy measurement.
The JVM start up can be avoided by starting the program without measuring anything and just start the measurement when the computation method is ready to be started and not for the whole machine. Avoiding reading JVM startup, warm up, even unnecessary computations, and focusing only on the necessary parts.
Garbage collector can be harder, while \texttt{System.gc()} can be used to suggest a garbage collection cycle, the JVM is free to ignore this request. As such, it is not a reliable mechanism to control GC timing. An alternative is to increase the memory the JVM can use, so the GC is called fewer times. 


It is also important to note that:


\begin{itemize}
    \item \textbf{Use multiple JVM invocations}: Run benchmarks in new JVM instances to capture variations and prevent biased warm-up effects.
    
    \item \textbf{Use multiple JVMs and versions}: Different JVM implementations and versions may apply different optimizations, affecting performance outcomes.
    
    \item \textbf{Avoid cherry-picking values}:  Instead of reporting only the best or worst results, use the median or mean along with variability.
    
    \item \textbf{Vary hardware configurations}: Using different hardware will have different results.
    
    \item \textbf{Test with multiple heap sizes}: Garbage collection behavior and performance can change depending on the heap size. Test different benchmarks with a range of heap sizes, starting from the minimum required.
    
    \item \textbf{Use realistic, diverse workloads}: Use real-world applications instead of microbenchmarks to better reflect practical behavior.
    
    \item \textbf{Avoid background system noise}: When running the benchmarks, preferably only have the necessary processes running, to reduce measurement noise.
\end{itemize}

When building the tool, it is necessary to gather data from the Java programs, which requires benchmarking best practices. Although it is not possible to apply all of them due to time and complexity constraints, it is always helpful to keep in mind what it takes to create a good benchmark. In this project, the techniques applied included using multiple JVM invocations, avoiding cherry-picking by reporting average values, minimizing background system noise, excluding JVM startup time, and focusing measurements on the relevant computation phases.



\section{Energy Profiling} \label{sec:background_energy_profiling}

To fully understand the processes involved in this work, it is important to first understand what energy profiling is, as it will be frequently referenced and used later.

Energy profiling is the systematic process of measuring, monitoring, and analyzing the power consumption of a system, using specialized software or hardware tools to collect detailed power consumption data. This data can be collected from different parts of a system, including applications, processes, and specific snippets of code, to provide insight into how different components and activities contribute to overall energy consumption. Through this type of power consumption profiling, developers are able to identify inefficiencies and optimize software to improve energy efficiency. This makes it very important for extending battery life in mobile devices, reducing operating costs in data centers, and also minimizing environmental impact through energy consumption. Energy profiling is a step toward making informed decisions to develop more sustainable and cost-effective computing solutions.

Energy profiling bridges the gap between traditional performance metrics and energy consumption, offering developers critical insights into how their software impacts system efficiency.
When programming, developers usually consider the time it takes to complete a program, or the response time from client to server, or the amount of memory it uses. Most don't have an idea of how much energy their program consumes, or how much it can consume in certain cases, and getting this idea is not as trivial as it seems. To get this awareness, measurements could be made, but they are also difficult to get compared to measuring the time a program takes, which is checking the differences in timestamps, or understanding the memory usage. As for measuring energy, the same program can produce different values due to its non-deterministic nature, meaning that results are often expressed as a range of possible values. Also, reducing the execution time of a program does not guarantee that power consumption will follow~\cite{10.1145/3136014.3136031}. To obtain the measurements it is necessary to use software based tools that can facilitate this process, sometimes at the cost of less accurate readings or hardware devices for accurate values.

To perform hardware-based measurements, a power monitor or power meter device~\cite{hackenberg2013power,ge2009powerpack} must be used to obtain the precise values that a system uses when connected to the electrical grid. These devices measure the power drawn from the grid to the machine, but they usually measure the power consumed by the entire system rather than by specific pieces of hardware, which means they have low granularity. Achieving granular measurements, such as isolating power consumption for specific components like the CPU or RAM, requires a more complex setup to avoid reading unnecessary power consumption. This approach is not suitable for developers who only want to analyze the energy performance of their programs, which shows the difficulty of measuring energy consumption.

An alternative with minimal overhead is to use software-based tools. These tools are typically easier to use, but may not provide values as accurate as hardware-based measurements. However, they are more versatile and can be implemented or modified to meet the user's needs. 


\section{Energy Tools} \label{sec:background_energy}

Understanding and optimizing energy consumption in software requires specialized tools capable of measuring and analyzing power usage. These tools provide valuable insights into how programs consume energy during execution, enabling developers to identify inefficiencies and optimize performance.

At the start some tools were tested in order to see the one that best suited the needs of the project:

Intel RAPL (Running Average Power Limit)~\cite{intel_rapl} is a tool for monitoring power consumption. It utilizes Model-Specific Registers (MSRs), which are used for program execution tracing, performance monitoring, and toggling CPU features. These registers can store the total energy usage of the CPU and memory, allowing it to be read and analyzed. Most software based tools rely on RAPL to measure energy consumption, since it is pretty accurate and is widely available in most CPUs.


Perf~\cite{perfwiki_main} is a Linux tool primarily designed for analyzing application performance characteristics rather than precise energy measurement. While it can provide some energy-related metrics, its measurements tend to be imprecise. In this context, Perf was used mainly to get a rough idea of energy consumption and to serve as an alternative when more accurate tools were unavailable.

Powertop~\cite{archlinux_powertop} was also tested, but it could only perform energy measurements on laptops, as it relies on battery drain data to calculate energy consumption. Since this approach does not align with our specific requirements, we considered Powertop as a last-resort option.

JoularJx~\cite{noureddine-ie-2022} is an energy measurement tool capable of measuring the consumption of Java programs and its methods. However, it is not as precise as other tools as it requires to measure the entire start of the JVM and whole functions instead of small code blocks.


PowerJoular~\cite{noureddine-ie-2022} is an open source tool, capable of measuring energy, from the CPU and GPU, using the Intel RAPL power data through the Linux powercap interface it can read the energy from the CPU and for the GPU it uses NVIDIA SMI to directly read the power consumption.
To read the power consumption of specific processes, PowerJoular monitors the CPU cycles and utilization of each process. By knowing the total power consumption of the CPU through the RAPL interface, it can calculate the power usage of individual processes based on their CPU utilization.
It is build in ADA, that is considered one of most energy efficient programming languages~\cite{PEREIRA2021102609}, and it can monitor applications by name or PID. In this work, it will be necessary to measure the energy consumption of programs, methods, and specific code snippets. To achieve this, PowerJoular will be employed as the primary tool for energy profiling.

Experiment-Runner~\cite{S2_Group_Experiment_Runner} is a framework built in python made to facilitate experiments, it is easily customizable and can be used with energy measurement tools, such as PowerJoular, to monitor the power consumption of another process. While it offers robust support for energy-related experiments, some issues were identified during its use, which will be discussed in detail in Section~\ref{chapter:results}.

\begin{table}[H]%[htbp]
\centering
\caption{Comparison of energy measurement tools}
\begin{tabular}{|p{2.8cm}|p{5.4cm}|p{5.4cm}|}
\hline
\textbf{Tool} & \textbf{Pros} & \textbf{Cons} \\
\hline
\textbf{Intel RAPL}~\cite{intel_rapl} & 
- Hardware-level accuracy \newline
- Widely supported in modern CPUs \newline
- Low overhead and high precision & 
- Not easily accessible \newline
- Cannot measure per-process energy directly \\
\hline
\textbf{Perf}~\cite{perfwiki_main} & 
- Standard Linux tool \newline
- Useful for performance + rough energy metrics & 
- Not designed for energy profiling \newline
- Low precision in energy consumption \\
\hline
\textbf{Powertop}~\cite{archlinux_powertop} & 
- Easy to use \newline
- Good for overall system power stats on laptops & 
- Relies on battery drain \newline
- Ineffective on desktops or servers \\
\hline
\textbf{JoularJx}~\cite{noureddine-ie-2022} & 
- Measures Java program energy \newline
- Method-level granularity possible & 
- Less precise \newline
- Only works with Java \newline
- Can only measure large code blocks \\
\hline
\textbf{PowerJoular}~\cite{noureddine-ie-2022} & 
- Support for CPU and GPU \newline
- Supports per-process measurement \newline
- Built in energy-efficient ADA \newline
- Monitors by name or PID & 
- Multi-thread measurement is limited \newline
- Slightly more complex setup \\
\hline
\textbf{Experiment Runner}~\cite{S2_Group_Experiment_Runner} & 
- Highly customizable \newline
- Integrates well with PowerJoular \newline
- Ideal for repeatable experiments & 
- Limited control over measurement internals, since it relies on a prebuilt framework instead of a custom solution \newline
- Some bugs and issues in practical use \\
\hline
\end{tabular}
\label{tab:tool_comparison}
\end{table}



\section{Code static and dynamic analysis} \label{sec:background_static_dynamic_analysis}

Static analysis, as the name implies, is the process of analyzing the code statically, meaning it examines the code without executing it. By examining the code, tools that leverage static analysis tools can understand how the program will behave at runtime~\cite{ernst2003static}, this analysis often aims for soundness, meaning that if the tool catches an error, it means that the error really exists, there are no false negatives. However, this can come at the cost of producing false positives, where issues that are not actually problems are flagged, so it's important to keeps a balance between them. This analysis allows to check the entire source code and every path, much like compilers check syntax and types. Still, they can only predict some behaviors, as some can only be found when the program is executed, for example, by using dynamic analysis.

Dynamic Analysis is the process of analyzing a program while it is running, allowing for real-time observation of its behavior.  This type of analysis leaves no doubt about memory usage, output, the path taken, how much time it took, among many other factors that are only clear when executing the program~\cite{ernst2003static}. A good example of dynamic analysis is unit testing, which tries to cover as many code paths as possible with different inputs, to understand as much as possible how the program works, and to find something that might be difficult to find with static analysis. However, because its nature, dynamic analysis can be time-consuming, it requires executing the program, which can be slow, especially for large applications or complex codebases. It also requires a lot of resources, such as memory and processing power, and it is not always possible to execute the program in all scenarios. For example, some programs may require specific hardware or software configurations that are not available in all environments. Additionally, dynamic analysis can only analyze the code that is executed during the run, which means that some parts of the code may not be covered by the analysis.
For fast power estimation, static analysis is often preferable due to its speed and scalability. While it may be less accurate than dynamic analysis, it remains a viable option, particularly for large codebases with many dependencies. Static analysis avoids the need for code execution and is generally less resource-intensive, as discussed by Ernst in the context of program analysis trade-offs~\cite{ernst2003static}.
In addition, static analysis is more portable because its setup is much simpler than the more complex setup required for dynamic analysis.
Developers may not have the time or the infrastructure to run the program just to get an average measure of energy consumption for a code snippet or program. Therefore, using static analysis to infer energy consumption makes sense in this context.

The use of static analysis implies the use of a parsing technique. This technique involves analyzing the syntactic structure of the provided code, respecting the rules of the language in which it is written. First, it is necessary to perform a lexical analysis to obtain the keywords, identifiers and tokens that the language contains. Then the parser uses these tokens together with the grammar rules of the programming language and outputs a tree. The output is an Abstract Syntax Tree (AST), which contains the logical structure of the code and allows further analysis. In the context of this work, this technique allows analyzing, for example, Java code and obtain its structure to find out which statements have been used.


The tool proposed in this work will primarily rely on static analysis to achieve its objectives. Static analysis, by examining the code without executing it, is particularly effective in providing early insights into potential energy inefficiencies during the development process. By evaluating all possible code paths and scenarios, it avoids the dependence on specific runtime conditions inherent in dynamic analysis, making it a powerful tool for identifying and addressing energy-related issues without the complexity of real-time monitoring.

However, to create the energy consumption dataset required to train machine learning models, this work will utilize dynamic analysis. This approach enables the collection of accurate and context-aware energy consumption measurements, which are crucial for building a reliable dataset to inform and enhance the energy optimization models. By combining the strengths of both static and dynamic analysis, this work aims to develop a comprehensive framework for energy-efficient software design.


\section{Language Server Protocol} \label{sec:background_lsp}

The Language Server Protocol (LSP) is an open protocol developed by Microsoft for separating language logic (such as code completion, diagnostic information, and symbol resolution) from the editor or development environment on which they are run. The objective of LSP is to allow programming language tools to be reused~\cite{10.1145/3550355.3552452} in different code editors to gain better portability while also reducing the work required to support different environments.

The LSP runs on top of a client-server architecture. The language server, which is usually implemented in the language being analyzed, provides a number of features, such as syntax highlighting, error detection, semantic analysis, and navigation in the source. The client, usually embedded in the editor, talks over the JSON-RPC-based protocol to the server~\cite{lsp}.

With LSP, language-specific behavior can be integrated in multiple editors without reimplementation of the underlying logic for each editor. For example, while in-plugin configuring of an environment such as IntelliJ IDEA or Eclipse generally requires deep integration of the in-plugin code with the specific platform’s native APIs, an LSP implementation eliminates such complexity. If the editor is LSP-compatible, something which is the case for most modern editors, the editor is then able to operate with the language server without further configuration. 

Within the context of the work, LSP is employed as the main technology for the development of a cross-platform language analysis backend. The language server is developed in Java, encapsulating the static analysis and model interaction logic, while the frontend is developed in web technologies (TypeScript and HTML) to provide the user interface within Visual Studio Code. 

While LSP provides cross-editor support for language-related features, it does not account for UI integration into editors. For example, VSCode supports custom interfaces like WebView APIs, which allow extensions to show HTML and TypeScript information in the editor. However, other IDEs like IntelliJ IDEA or Eclipse use different UI integration schemes and don't have web-based interfaces supported in the same way. Therefore, while the language analysis component of the extension can be reused in editors supporting LSP, the UI may need to be redesigned or adapted for each environment.


\section{Machine Learning} \label{sec:background_machine_learning}

Using a machine learning (ML) model to estimate energy consumption can offer advantages over a traditional approach. While traditional approaches such as empirical estimates based on historical data may work, they may not be the best solution in this case due to the unpredictable behavior of energy. Using an ML algorithm can help identify more complex patterns and provide a highly accurate estimate while adapting to the arrival of new information.

To predict energy, an ML model will be used. It's important to understand how different ML algorithms work and which ones are best suited for the proposed project. There are several ML algorithms, and they fall into four main categories~\cite{sarker2021machine}: supervised learning, unsupervised learning, semi-supervised learning and reinforcement learning.

The supervised ML requires labeled data for the model to train on. During training, the model has access to the input and output parameters, and it will try to match inputs to the correct outputs. It has two categories: classification, where it predicts discrete labels, such as whether a picture is a cat or a dog; and regression, where it predicts continuous values, for example, predicting the price of a house based on location, size, and so on, or, as in our project, predicting energy consumption. These models can be very accurate, but they can also make incorrect predictions for patterns they were not trained on. These techniques also are accompanied by some metrics to evaluate how well the models perform.



For \textit{classification models}, some of the most important metrics include:
\begin{itemize}
    \item \textbf{Accuracy:} Measures the percentage of correctly classified instances.
    \item \textbf{Precision:} Focuses on the correctness of positive predictions.
    \item \textbf{Recall (Sensitivity):} Evaluates how well the model detects actual positives.
    \item \textbf{F1 Score}: Balances precision and recall, useful when data is imbalanced.
    \item \textbf{ROC-AUC (Receiver Operating Characteristic - Area Under Curve):} Assesses classification performance across different thresholds.
\end{itemize}

On the other hand, some of the most important metrics for \textit{regression models} are:
\begin{itemize}
    \item \textbf{R² (R-squared or Coefficient of Determination):} Measures how well the model's predictions fit the actual data. A value close to 1 indicates a better fit.
    \item \textbf{MSE (Mean Squared Error):} Computes the average squared difference between predicted and actual values, penalizing large errors more heavily.
    \item \textbf{RMSE (Root Mean Squared Error):} The square root of MSE, offering a more interpretable error magnitude.
    \item \textbf{MAE (Mean Absolute Error):} Calculates the average absolute difference between predictions and actual values, making it less sensitive to extreme outliers.
\end{itemize}


In unsupervised ML, the model attempts to find patterns and relationships in the unlabeled data set. With this technique, it is possible to find similarities or clusters in the data, for example, to detect an anomaly in the data. This technique does not require the effort of acquiring labeled data, but also it will be harder to understand if the output is correct or not.

Semi-supervised learning, as the name implies, uses a combination of supervised and unsupervised learning techniques. This hybrid approach is very useful in real-world scenarios where parts of the data can be labeled and others can't, allowing for better predictions in the output.

Reinforcement learning, where the model receives different feedback for different tasks and uses this feedback to perform the tasks in the most optimal way. The feedback can be in the form of rewards or penalties so that the model can better understand whether it is doing the task correctly. For example, a model playing a video game is rewarded for completing levels faster and penalized for failing the level. This method of learning allows for complex solutions to sequence-based problems, such as robotics or gaming. However, it can be time-consuming and computationally expensive.

There are some algorithms that can meet the proposed model requirements, such as: Linear regression, Tree-Based Models (e.g., Random Forest, Gradient Boosting), Neural Networks, Gaussian Processes, Support Vector Machines (SVM) and or leveraging Genetic Programming.

The most common approach in ML is to use a linear regression algorithm, which is the simpler to implement and very effective. It can predict a continuous output based on the input independent variables. Linear regression is computationally efficient, easy to implement, and works well when the relationship between the input features and the output is linear. However, its simplicity is also its limitation, as it struggles with nonlinear relationships and can underperform when the input features interact in complex ways. It is also sensitive to outliers, which can significantly skew the results.

Tree-based models rely on decision tree models, which are used for structuring decisions, where in each branch a decision is made based on some criteria, and the end of the branch contains the final output. Random forest is a tree-based model that combines multiple decision trees to build an accurate model. When a prediction is needed, all the decision trees provide a vote, in classification or an average in regression and the random forest combines them to give the final prediction. Each tree is trained in different subsets of the data. This algorithm provides high accuracy, robustness to overfitting and estimates the features' importance, however, it has a higher computational cost, more memory usage, and it can take more time to reach a prediction than other approaches.

Gradient boosting is also a tree-based model, it builds trees (weak learners) sequentially with each of them trying to correct the errors of the previous one. It starts with a simple model and iteratively adds trees to reduce residual errors from the previous trees. It is generally more accurate than a random forest, however it is more prone to overfitting if there is a lot of noise in the data.

Neural networks models are particularly good at capturing complex, nonlinear relationships between inputs and outputs. A neural network consists of an input layer, hidden layers, and an output layer. The input layer receives the features (e.g., program attributes derived from the AST), the hidden layers process these features using weights, biases, and activation functions, and the output layer provides the final prediction. Neural networks are highly flexible and can adapt to various problem domains, automatically learning feature representations without requiring extensive manual engineering. However, they require large datasets to avoid overfitting and significant computational resources for training.

Gaussian Processes are particularly interesting, because they take into account the probabilistic nature of energy measurement and provide a range of possible values alongside with the probabilities. This makes them ideal for tasks where understanding the uncertainty in predictions is important, such as energy modeling. However, their computational complexity grows significantly with the size of the dataset, making them less practical for large-scale problems.

Support Vector Machines (SVMs) are a powerful tool in machine learning, capable of performing both classification and regression tasks. This algorithm identifies the optimal hyperplane in an N-dimensional space where it can separate all the features. When it is difficult to separate the features, a technique called kernel trick can be used to create an additional dimension to help separate them. This makes them good for high dimensional spaces, the ability to handle nonlinear relationships and the ability to ignore outliers. However, it can be harder to train this model, and tune the parameters.

Another alternative is to use genetic programming, which is not exactly a conventional ML algorithm, but rather a technique that can be applied to solve ML problems. It is a form of artificial intelligence inspired by the process of natural selection and evolution, where potential solutions to a problem are represented as programs or symbolic expressions. These programs improve iteratively, over generations, through mutations and crossovers, that sometimes can be random, guided by a fitness function. Genetic programming is useful for tasks like symbolic regression and feature classification. However, it can be computationally expensive and can produce inconsistent results due to its uncertain nature.

A good example of genetic programming in machine learning is Python Symbolic Regression (PySR). PySR builds on this technique to discover mathematical expressions that model data in a way that is both accurate and interpretable. Unlike black-box models, PySR generates human-readable formulas, making the results more interpretable and transparent.
PySR works by searching the space of mathematical expressions to find those that best fit the data:

\begin{itemize}
    \item It starts with a random population of simple mathematical expressions, such as $x + 1$ or $\sin(x)$.
    \item Evaluates how well each expression fits the data using a loss function (like MSE).
    \item Expressions are then evolved over several generations using operations like mutation and crossover to produce new, often more complex formulas.
\end{itemize}

At the end of the process, PySR provides a list of candidate expressions, typically sorted along a Pareto front balancing complexity and accuracy. While more complex expressions often provide better predictions, users can select simpler ones if interpretability is a priority.

This technique is particularly well-suited for energy prediction, as it can reveal easy-to-understand formulas that explain how certain parts of a program contribute to energy consumption. This can help developers understand and optimize the energy efficiency of their code more effectively than with traditional black-box models.

These algorithms were taken into account when developing the model that can best predict energy.
Specifically, the models evaluated in this project include: Decision Tree Regressor, Random Forest, Gradient Boosting, Linear Regression, and PySR. These algorithms were chosen based on characteristics such as the simplicity of linear regression, the predictive strength of trees, and the interpretability of symbolic regression with PySR. Together, these models offer a balance of accuracy, efficiency, and explainability, which is important for modeling energy consumption in Java applications.




% Estado da arte

\chapter{Related Work}\label{chapter:related_work}

This chapter discusses relevant approaches to energy measurement in software, its importance, and how to achieve it effectively. The chapter is divided into three categories: general context and approaches, static analysis-based tools, and dynamic analysis-based tools. This structure allows for a clearer comparison of technical strategies, their strengths, and their limitations.

\section{General Context and Existing Approaches}

Energy efficiency is a critical focus across industries, as it directly impacts global sustainability, economic costs, and product quality.  The goal is to reduce greenhouse gases to create a sustainable future, reduce infrastructure costs, and improve product quality~\cite{annurev:/content/journals/10.1146/annurev.resource.102308.124234}. 

In particular, large scale computation and communication consume a lot of global energy, and these values have been increasing in the last decades, so the topic of energy aware programming and energy efficient software has been targeted by many researchers in recent years with the objective of reducing energy costs in large IT infrastructure~\cite{8880037}.
This improvement can be considered an optimization problem and can be tackled in several ways for example a heuristic approach by adjusting the hardware performance dynamically, or completing tasks in their deadlines, using the least energy possible. However, some of these implementations can only be short term solutions and in long term, the focus will be toward more complex models that can predict and optimize performance relative to hardware configurations~\cite{10.1145/1666420.1666438}. 

An approach to increasing developer's awareness of the energy consumption of their code involves creating extensions to already used programming languages, such as Java. For example, ECO~\cite{7194624}, a programming model as a minimal extension of Java. By rewriting some parts of the code to this extension syntax it is possible to define resource limits on the battery or temperature implementing adaptive behaviors through modes, and leveraging runtime monitoring.

In addition, new languages can be developed to address these goals, as demonstrated by ENT~\cite{10.1145/3062341.3062356}. ENT is a Java extension that empowers programmers with more direct control over the energy consumption of their applications. ENT’s type system enables applications to adapt dynamically to power constraints by switching operational modes based on resource availability, such as battery level or CPU temperature, allowing for software-level energy optimization. However, the language introduces complexity, making it potentially challenging for developers to learn and adapt to existing codebases. 

Using machine learning algorithm has also shown to be effective to estimate energy consumption. Fu et al.~\cite{8726531} used four distinct ML algorithms (Ridge Regression, Linear Regression, Lasso, and Random Forest) to analyze the energy consumption of various apps, achieving low average error rate. These findings demonstrate the potential of such models to serve as the foundation for future tools, enabling developers to predict and optimize software energy usage without relying on specialized hardware. Other works go even further with using machine learning for energy efficiency and explore LLMs for generating not only optimized code but even reviewing and refactoring software for sustainability. Rani et al.~\cite{rani2025can} demonstrates that while LLMs recommend a range of optimizations, from vectorizing to memory management, their effectiveness in terms of actual energy savings remains low with LLMs usually making trade-offs between memory and execution time. Cappendijk et al.~\cite{cappendijk2024generating} investigate prompt-based optimizations by modifying LLM inputs with sustainability-focused instructions. While some prompt–model–task combinations achieved up to 59\% reduction in energy consumption, the results were highly inconsistent, with some prompts even increasing energy usage by over 400\%. Their findings underline both the potential and current limitations of using LLMs to directly influence software energy efficiency through prompt engineering. 


Estrada et al.~\cite{estrada2022learning} proposed an energy consumption prediction model to optimize energy management in cloud and fog infrastructures, addressing challenges such as high operational costs and environmental impact. Their system integrates machine learning with sensor-based hardware to create a non-intrusive monitoring approach. Using a network of sensors to collect real-time data on metrics like voltage and power, processed via MQTT and visualized on dashboards, the study employed a robust linear regression model to predict hourly energy consumption. The research emphasizes the importance of real-time monitoring and machine learning integration for achieving energy efficiency in data centers, aligning with Green IT principles.

\section{Static Analysis Tools}

The use of static analysis can be valuable for understanding how instructions affect the energy consumption of programs. Aggarwal et al.~\cite{aggarwal2014power} shows that system calls are directly related to energy consumption in Android applications. With this insight, it's possible to use static analysis to identify system calls within the code. This information can then be used to infer potential energy usage patterns, providing an early indication of where higher energy consumption may occur. This approach highlights the importance static analysis can have to understand program energy behaviors.

To tackle the problem of energy consumption in IT, some solutions have been presented. Some researchers focused on using energy measurement tools, like JRAPL to measure common libraries in Java and understand how much energy they use and what are the best alternatives to improve the energy efficiency of the code~\cite{10.1145/2896967.2896968}. Observing common libraries for the implementation of list, sets and maps, is possible to see which ones have the better energy efficiency and what changes could improve the code.
Hasan et al.'s~\cite{10.1145/2884781.2884869} research adds to this by creating detailed energy profiles for various Java collection classes, including lists, maps, and sets, across different implementations (Java Collections Framework, Apache Commons Collections, and Trove). Their work presents concrete quantification of energy consumption in these collections based on common operations such as insertion, iteration, and random access, and highlights the performance impact of collection types on energy efficiency for different input sizes.

However, because these collections are often used with threads, it is important to understand how much energy efficiency can be improved without compromising thread safety. The energy consumption of Java's thread-safe collections was studied by Pinto et al.~\cite{7816451}, where researchers demonstrated that switching to more energy-efficient collection implementations can reduce energy usage while maintaining thread safety.

Building on these efforts, Pereira et al.~\cite{10.1145/3238147.3240473} introduced a static analysis tool (Jstanley), as part of an Eclipse plugin, that can detect energy inefficient collections and recommend better alternatives. While Jstanley demonstrated notable improvements in energy efficiency within its specific context, it has several limitations.
For example, they only account for three collections interfaces, (Lists, Sets and Maps) and only account for three different sizes (25,000, 250,000 and 1,000,000), it does not account for loops, thread safe and thread unsafe collections. Compared to our approach, Jstanley is limited, as it does not provide the actual information about the energy spent, it just shows recommendations. While the tool shows great improvements in its tested environment, replacing collections may not be enough in many practical cases. A more extensive tool, capable of analyzing a wider range of collections and providing energy metrics, would enable developers to achieve even greater energy efficiency and awareness.

Oliveira et al.~\cite{8816747}, proposed a tool (CT+) that is capable of performing static analysis of the code, recommend changes that reduce energy consumption and automatically apply these changes. It is achieving up to 16.34\% reduction in energy consumption even for a real-world, mature system such as Tomcat, it could reduce the energy consumption about 4.12\%. It improved from previous works by taking into account more collections implementations, more operations, thread safety and support for mobile applications. Although CT+ is a significant improvement over previous tools, it still has some limitations. For example, it does not provide the actual energy consumption of the code, it just shows recommendations based on the previously collected statistics, as it did not leverage machine learning. Additionally, it only works with collections, and it is not easily generalizable for other context. 

\section{Dynamic Monitoring and Profiling Tools}

In addition, SEEP~\cite{10.1145/2094091.2094106} uses symbolic execution for energy profiling, generating multiple binaries representing different code paths and input scenarios. By analyzing these binaries with hardware-based energy measurement devices, SEEP provides energy consumption data, offering a deeper understanding of code efficiency across various inputs and paths. This approach complements other tool, called PEEK~\cite{187026}, which builds on SEEP to help developers optimize energy usage with minimal effort. PEEK is an IDE-integrated framework that guides developers in writing energy-efficient code. It has a front end for IDE interfaces (e.g., Eclipse, Xcode), a middle end to manage data and versioning via Git, and a backend where energy analysis is performed, either through SEEP or hardware devices. Through these layers, PEEK identifies inefficiencies and suggests optimizations, supporting efficient coding practices. However, it has some limitations when compared to the proposed approach in this work, it uses dynamic analysis instead of static analysis, which was already explained in Section~\ref{sec:background_static_dynamic_analysis}, why it was chosen over dynamic analysis. Additionally, it does not incorporate any machine learning techniques.

Some command tools, that work on Linux, help facilitate the process of energy measurement, like Perf~\cite{perfwiki_main}, that is a command line tool already available in Linux, and is mainly used for performance monitoring and profiling. Although it's not specific for energy measurement, it can do it, with Intel RAPL but not as practical as other tools, specially when it's needed to measure a single process energy consumption. Powertop~\cite{archlinux_powertop} is another tool capable of providing the power consumption, however it only works for laptops, as it requires to check the battery to see how much energy was used and calculate the power consumption.

JoularJx~\cite{noureddine-ie-2022} is a Java agent that attaches to the Java Virtual Machine (JVM) at startup to monitor energy consumption. It runs in a separate thread, collecting CPU usage data for the JVM, its threads, and individual methods using statistical sampling. JoularJx performs power estimation by using platform-specific tools, such as the Intel API through the Linux RAPL interface, or a custom regression model for Raspberry Pi devices. It periodically analyzes stack traces to isolate energy consumption at the method level, taking into account execution paths and separating application-specific calls from system or agent-induced calls.

\section{Limitations of Existing Approaches}

The reviewed solutions showed strong potential in energy-aware software development, showcasing effective strategies such as including programming languages extensions, machine learning models, the use of static and dynamic analysis. However, these solutions have some limitations, like the need to execute the code before showing the average energy cost to the developer, having limited collections or lacking integration with the development workflow. These limitations point to the need for a more accessible, less limited solution.

The objective of this work aimed to develop a process that could obtain models that statically estimate program energy consumption, by combining established best practices with unique, domain-specific techniques. The final models produced by the framework pipeline can be trained using various language-specific collection libraries, and they are flexible enough to be modified or enhanced as needed. The final extension tool will then display the energy consumption of the code being analyzed making the developer more aware of the program's energy usage and helping him make better decisions.


% methodology
\include{./tex/methodology}

% trabalho realizado (podem ser vários capítulos)
\chapter{Results}\label{chapter:results}

%\section{The issue with inconsistent energy readings} \label{sec:results_the_issue_with_inconsistent_energy_readings}

In this section we will report the results of the experiments conducted to evaluate the framework and its components. The results will be presented in a structured manner, focusing on the performance of the program generator, the orchestrator, the model training, and the extension tool. Each section will provide insights into the effectiveness of the framework in measuring and predicting energy consumption in Java programs.

Our hardware infrastructure is composed of two systems, one high-performance workstation used for data generation and collection, and a lower-spec machine used for testing the extension tool. The high-performance workstation is equipped with an AMD Ryzen Threadripper 3960X CPU, 94 GiB of RAM, and an NVIDIA GeForce RTX 3090 Ti GPU, while the lower-spec machine has an AMD Ryzen 5 3600 CPU, 16 GiB of RAM, and an NVIDIA GeForce RTX 3060 Ti GPU. The details of the hardware specifications can be found in Table~\ref{tab:hardware_specs}.

\begin{table}[htbp]
  \small
  \centering
  \caption{Hardware Infrastructure Specifications}
  \label{tab:hardware_specs}
  \begin{tabularx}{\textwidth}{l X}
    \toprule
    \textbf{Component} & \textbf{Specification} \\
    \midrule
    \multicolumn{2}{l}{\textbf{System 1 — Data Generation and Collection (High-Performance Workstation)}} \\
    \cmidrule(r){1-2}
    OS & Ubuntu 24.04.2 LTS (x86\_64) \\
    CPU                 & AMD Ryzen Threadripper 3960X (24 cores / 48 threads). 2.2 GHz, up to 5.05 GHz \\
    RAM                 & 94 GiB \\
    GPU                 & NVIDIA GeForce RTX 3090 Ti (GA102) with 24 GiB VRAM \\
    \midrule
    \multicolumn{2}{l}{\textbf{System 2 — Machine for Testing (Lower-Spec)}} \\
    \cmidrule(r){1-2}
    OS                  & Ubuntu 24.04.2 LTS (x86\_64) \\
    CPU                 & AMD Ryzen 5 3600 (6 cores / 12 threads). 3.6 GHz, up to 4.2 GHz \\
    RAM                 & 16 GiB \\
    GPU                 & NVIDIA GeForce RTX 3060 Ti with 8 GiB VRAM \\
    \bottomrule
  \end{tabularx}
\end{table}


The more powerful system, System 1, as described in Table~\ref{tab:hardware_specs}, was used for program generation and data collection, as it was configured to work with SLURM (Simple Linux Utility for Resource Management). SLURM is a highly scalable, open-source job scheduler used to efficiently manage compute resources on shared systems. It allows tasks to be queued and scheduled based on resource availability and job priority, helping to organize workloads across users without manual intervention. The model training did not use SLURM, as it did not require significant computation capabilities.

At the beginning of the experiment, some tools (described in Section \ref{sec:background_energy}) were tested to observe their behavior, understand how to use them, and determine which one best suited the needs of the project.

During the initial testing, Experiment Runner~\cite{S2_Group_Experiment_Runner}, a Python framework designed to facilitate experiment measurements was our tool of choice. The framework included an initial test template that used PowerJoular to measure energy of programs. While using the template and testing the framework some bugs and unexpected results were found, some of which were due to misuse of the framework.
Due to these issues, a tailor-suited Java orchestrator was developed. Although it performed the same core function as Experiment Runner (measuring energy consumption), it was more straightforward, focusing exclusively on energy measurement rather than providing a general-purpose solution.

However, some discrepancies were observed between the energy values measured by the Java and Python frameworks. This was unexpected, as both used the same energy measurement tool (PowerJoular) and measured the same program in the same way. Still, the Python framework consistently reported lower energy values than the Java version.
To investigate which tool was causing the inconsistency, \st{two} \wo{three} more orchestrators were implemented, including a simplified Python version. In total, four orchestrators were developed: Java, Python, C, and Bash. All four performed the same process, calling PowerJoular to measure the energy usage of a Java program.
This setup represents an early iteration of the approach later detailed in Section~\ref{sec:work_stage2_orchestrator}, which utilized signal-based control to ensure that the measurement tool only ran during the exact computation period being measured.

Figure~\ref{fig:4_orchs_comparison} presents the used CPU power observed by the four orchestrators while running 
 100 times a Fibonacci recursive program written in Java. They are ordered by the less energy to the highest energy. \st{And it shows the energy reads for the four different orchestrators used.} The labels contain the average energy values and its standard deviation.

\begin{figure}[htbp]
  \centering
  \includegraphics[width = .8 \textwidth]{figures/4_orchs_comparison.pdf}
  \caption{Orchestrators Comparison}
  \label{fig:4_orchs_comparison}
\end{figure}

It is noticeable that the Python orchestrator read energy values lower than the other orchestrators. Further analysis of the orchestrators revealed a notable difference in behavior. When the Python orchestrator was running, both the parent and child processes consumed CPU resources. In contrast, the other orchestrators (Java, C, and Bash) showed CPU usage only in the child process. This disparity may explain why PowerJoular reported lower energy consumption for the Python orchestrator. Since the CPU load was shared between the parent and child processes, PowerJoular, which measures energy only for the child process (in this example, the target Fibonacci program), captured less total energy usage.
Since the experiment runner included an example demonstrating how to use the framework with PowerJoular, the authors were made aware of this potential conflict when launching PowerJoular from Python, {\color{blue}having been warned of this potential conflict via email}.


\section{Models Metrics} \label{sec:models_metrics}

To instantiate the tool, models were trained on methods from the Java collections framework, specifically focusing on the \texttt{List} and \texttt{Map} interfaces. Data for these methods was generated, collected, and used to train the models. With the models in place, it is now possible to evaluate their performance using metrics such as R² and Mean Squared Error (MSE), based on the features extracted during data collection.

%\begin{figure}[htbp]
%  \centering
%  \includegraphics[width=0.85\textwidth]{figures/r2_comparison.pdf}
%  \caption{R² Comparison (\textit{Higher is better})}
%  \label{fig:r2_comparison}
%\end{figure}

Figure~\ref{fig:r2_comparison} presents the R² of all the methods that were analyzed from the \texttt{List} and \texttt{Map} interfaces. A score of 1 would mean that the model can get 100\% of the prediction right\. However, most of the time, it is not feasible to achieve a score of 1. In fact, most values are really low (bellow 0.5), which means that the model cannot predict the energy very well for some methods. The best models were for the method \texttt{addAll()} and \texttt{containsAll()} of the \texttt{List} collection, which got an R² of around 0.8 for most models.

\begin{figure}[htbp]
  \centering
  \makebox[\textwidth][c]{%
    \hspace{5mm}\includegraphics{figures/r2_comparison.pdf}
  }
  \caption{R² Comparison (\textit{Higher is better})}
  \label{fig:r2_comparison}
\end{figure}

These three methods with the highest accuracy were the ones that generated bigger energy outputs, because the methods were computationally more intenstive than the others. This made it so that bigger inputs would result in bigger energy outputs, and make the models more easily find a pattern to predict the energy. As for the other models, since a bigger input would not mean a bigger energy output, the model, might have some difficulties predicting the energy. This does not mean that a lower R² will completely make the model unusable for some methods, as their MSE, that can be seen in the Figure~\ref{fig:mse_comparison} is not high, meaning that even when failing to predict the energy of the method, the failed prediction will not be far away as one might expect. For example, if a model has an MSE of $1 \times 10^{-10}$ and predicts the energy usage to be $1~\mathrm{J}$, then even if the prediction is not exact, the true value is likely within $\pm \sqrt{1 \times 10^{-10}} = \pm 1 \times 10^{-5}~\mathrm{J}$ of the prediction, indicating accurate performance despite a potentially low R². 

\begin{figure}[htbp]
  \centering
  \makebox[\textwidth][c]{%
    \hspace{5mm}\includegraphics{figures/mse_comparison.pdf}
  }
  \caption{MSE Comparison (\textit{Lower is better})}
  \label{fig:mse_comparison}
\end{figure}

Another factor that may contribute to low R² scores is the behavior of the energy measurement tool when applied to certain methods. For instance, some methods, such as \texttt{size()}, do not require more computational effort as the collection size increases. Whether the collection has one element or one million, the method executes in roughly the same amount of time. Consequently, the energy consumption remains nearly constant, regardless of the input size. Since these methods complete \st{very} quickly and consume \st{very} little energy, even minimal measurement noise can significantly affect the recorded values. Figure~\ref{fig:size_energy} illustrates this effect: although an increase in energy with input size is typically expected, the recorded energy values for \texttt{List.size()} remain considerably flat. While this example isolates a single feature, and other features also influence energy consumption, input size is often the most significant. This observation suggests that for low-energy, low-variance methods, measurement noise can impact the signal, making accurate energy prediction particularly challenging.

\begin{figure}[htbp]
  \centering
  \includegraphics[width = .8 \textwidth]{figures/size_energy.pdf}
  \caption{Energy for the \texttt{List.size()} method with different list sizes.}
  \label{fig:size_energy}
\end{figure}


Nonetheless, the lower R² scores should be addressed, and what can be done to improve the results of these models is to have the program generator create higher inputs, so the energy profiles also have outputs with higher energy, making the energy predictions more accurate. Then, since most of the predictions depend on the input, it means the other features do not have such higher impact, so it is also important to pick better features and remove others that might not be interesting. {\color{blue}It is possible to improve a complex method that relies on a simpler method by incorporating the features or predictions of the simpler method, which can help improve the accuracy of the model for the complex method. In particular, accessing the energy profiles of simpler methods may refine the R² score of the complex method, as part of its behavior can be decomposed into more predictable components.}



In general, most of the models present similar scores, however we used PySR in the instantiation. The biggest advantage over the other models is that it can represent the predictions in expressions which can help the users to try to understand why the code is using more energy. It has a nice feature of allowing to balance complexity and accuracy. And can easily be used in another code language as it is represented as a mathematical expression.
The fact that most models present similar results, ranging from advanced models like Random Forest and Gradient Boosting to simple models like Linear Regression, suggests that features used in the model likely do not capture highly nonlinear or complex relations. This indicates that energy consumption behavior in analyzed approaches can be reliably captured by simple relations. As a result, the set of features cannot offer the richness or variability needed for distinguishing more subtle energy behavior. This does not mean that the learning problem is simple, rather, it indicates that the available features may fail to express potential nonlinear patterns of energy consumption. Although the strategy should be enough for most cases, to improve prediction performance and enable more effective use of expressive models, future work should incorporate features capable of highlight the complexity of predicting energy usage.



\section{Predicted vs. Measured Energy} \label{sec:predicted_vs_measured_energy}

Leveraging the developed models and the extenstion, we can present the results and evaluate how accurately the tool estimates energy consumption compared to real measurements obtained using PowerJoular. Since the energy profiles were generated on a specific hardware setup (see Table~\ref{tab:hardware_specs}), the resulting models are adjusted to that particular system. If the tool were run on a different machine, the absolute energy values would likely differ due to variations in hardware characteristics. However, the relative differences in energy consumption between operations are expected to remain consistent. For instance, if \texttt{TreeMap} consumes more energy than \texttt{HashMap} on one system, the same trend will likely hold on another system, even if the exact energy values vary.

\begin{listing}[H]
\noindent\rule{\linewidth}{0.4pt}
\begin{minted}[linenos, fontsize=\small, frame=none, bgcolor=white,breaklines=true,breakanywhere=true]{Java}
    ArrayList<Integer> l = new ArrayList<>();
    ArrayList<Integer> l2 = new ArrayList<>();
    l.addAll(l2);
\end{minted}
\noindent\rule{\linewidth}{0.4pt}
\caption{Code example}            
\label{lst:code_example}
\end{listing}

To test the extension tool accuracy, the energy measurement needs to be run in the same machine were the data was collected. It is also important to take into account that the measured energies from these experiments were not processed exactly like the orchestrator, as the orchestrator processer has a more general approach that can introduce overhead in some cases. {\color{blue}So the experiments performed, when possible, were only measuring the exact targeted method. For example, to measure the method \texttt{foo(type)} the orchestrator would need to create an array of parameters and loop through it during 1 second or until the end of the array size. However, during these experiments, if a single invocation of the method was sufficient for PowerJoular to capture its energy consumption, then only that single call was performed. This minimizes overhead and more accurately reflects real-world usage scenarios}.

For this a simple program was developed, Listing~\ref{lst:code_example}, more like simple Java instructions, of just creating two lists (both with size 1000) and using the method \texttt{addAll(Object)} and checking if the prediction matches the actual measurement.

\begin{table}[htbp]
  \centering
  \label{tab:energy_comparison}
  \footnotesize
  \begin{tabular}{>{\raggedright\arraybackslash}p{4cm}ccc}
    \toprule
    Method & Actual Energy Range (J) & Predicted Energy (J) & Prediction Error (\%) \\
    \midrule
    \texttt{addAll(Object)} & 4.9e-4 -- 5.1e-4 & 5.51e-4 & 10.2 \\
    \midrule
    \texttt{addAll(Object)} + \texttt{size()} + \texttt{equals(Object)} & 5.0e-4 -- 5.2e-4 & 6.49e-4 & 27 \\
    \midrule
    \texttt{Map.put(Object, Object)} + \texttt{loop(1000)} & 0.0027 -- 0.0030 & 0.033 &  1058 \\
    \bottomrule
  \end{tabular}
  \vspace{0.5em}
  \caption{Comparison of actual and predicted energy consumption for different methods}
  %\footnotesize{The actual energy ranges were obtained by running the program 10 times.}
\end{table}


%\begin{table}[htbp]
%  \centering
%  \footnotesize
%  \makebox[\textwidth][c]{%
%    \begin{tabular}{@{}p{5cm}@{\hspace{2.5em}}c@{\hspace{0.5em}}ccc@{}}
%      \toprule
%      Method & Actual Energy Range (J) & Predicted Energy (J) & Input Value (\%) & Prediction Error (\%) \\
%      \toprule
%      \texttt{BinaryTrees.createTree(int)} & 3.75e-4 -- 4.60e-4 & -0.083 & 10 & 19980 \\
%      \midrule
%      \texttt{BinaryTrees.createTree(int)} & 11.78 -- 12.63 & 10.26 & 23 & 15.9 \\
%      \midrule
%      \texttt{BinaryTrees.createTree(int)} & 77.29 -- 80.63 & 79.90 & 26 & 1.2 \\
%      \midrule
%      \texttt{BinaryTrees.checkTree(TreeNode)} & 1.64e-4 -- 1.75e-4 & 1.88e-4 & 10 & 10.9 \\
%      \midrule
%      \texttt{BinaryTrees.checkTree(TreeNode)} & 1.37 -- 1.56 & 2.19e-3 & 23 & 99.85 \\
%      \midrule
%      \texttt{BinaryTrees.checkTree(TreeNode)} & 11.72 -- 13.29 & 3.15e-3 & 26 & 99.97 \\
%      \midrule
%      \texttt{BinaryTrees.trees(int)} & 0.041 -- 0.045 & -3.42 & 10 & 127 \\
%      \midrule
%      \texttt{BinaryTrees.trees(int)} & 713 -- 800 & 517.89 & 23 & 31.59 \\
%      \midrule
%      \texttt{BinaryTrees.trees(int)} & 6260 -- 7111 & 2429 & 26 & 63.66 \\
%      \midrule
%      \texttt{BinaryTrees.checkTree(TreeNode) + createTree(int) + trees(int)} & 4.59e-2 -- 4.77e-2 & -3.49 & 10 & 7578 \\
%      \midrule
%      \texttt{BinaryTrees.checkTree(TreeNode) + createTree(int) + trees(int)} & 796 -- 806 & 528.15 & 23 & 34.08 \\
%      \midrule
%      \texttt{BinaryTrees.checkTree(TreeNode) + createTree(int) + trees(int)} & 7062 -- 7350 & 2509 & 26 & 65.18 \\
%      \bottomrule
%    \end{tabular}%
%  }
%  \caption{Comparison of actual and predicted energy consumption for BinaryTrees program}
%  \label{tab:energy_comparison_bin_trees}
%\end{table}


The actual value measured by PowerJoular is in the range of 4.9e-4 to 5.1e-4J while the prediction is 5.51e-4J. This shows that the prediction (about 10.2\% prediction error) was good for this method in particular.
When trying to add two more operations (\texttt{size()} and \texttt{equals(Object)}) the measurement is in the range of 5.0e-4 to 5.2e-4J and the prediction is 6.49e-4J, about (about 27\% prediction error). Usually, the method \texttt{addAll(Object)} is one of the methods that has the best accuracy of around 80\%. Composing \texttt{addAll(Object)} with the other methods (\texttt{size()} and \texttt{equals(Object)}) that presented a low R² creates a more complex behavior and, as expected, increases the prediction error. Although the prediction is not as accurate as the previous one, it still provides a reasonable estimate of the energy consumption.

However, if we use the program presented in Listing~\ref{lst:Java_program_to_count_word_frequencies_in_a_string}, which involves a Map and a loop, the prediction accuracy drops significantly. The actual energy measured is in the range of 0.0027J to 0.0030J, while the prediction is 0.033J, resulting in a prediction error of about 1058\%. This discrepancy highlights the limitations of the model when applied to more complex operations involving Maps and loops.

The models trained have a low accuracy for Map methods. The prediction error is very high, which means that the model is not able to predict the energy consumption of these methods accurately. This is likely due to the fact that the models were trained on a limited set of data and do not generalize well to more complex operations.

This issue also highlights a limitation of our instantation: by composing with methods that have low R² scores, the prediction accuracy is significantly affected. The models trained on methods with low R² scores struggle to accurately predict energy consumption for more complex operations, leading to high prediction errors, which leads to an error accumlation problem. Potential solutions could involve introducing a correction factor or calibration step based on empirical error measurements, adjusting the final predicted value to better reflect observed trends.

{\color{blue}The results demonstrate that the extension tool is capable of providing useful estimations, especially in terms of relative trends between methods and input sizes. Proving it is capable of helping software developers on being more aware of the energy consumption of their codes. It shows potential in prediction, especially, for methods that are more impacted by the input. However, it is noteworthy that not all methods had the same high prediction accuracy as \texttt{addAll(Object)}, and combining multiple methods or increasing program complexity tends to amplify prediction errors.
These limitations highlight the importance of carefully selecting input sizes and operations for meaningful energy analysis, and they suggest that the tool is best suited for relative comparisons rather than precise energy estimation.}

\wo{eu não gosto muito do tom aqui, parece que estás a dizer que o tool tem mais problemas do que solução. Acho que precisamos falar das limitações, claro, mas precisamos dar ênfase ao que fazemos bem. Seria bom reescrever isto para que o tom seja mais positivo, e que se foque mais no que conseguimos fazer bem, e menos no que não conseguimos fazer. Eu percebi que em todos os casos que tu mostrou os erros foram que superestimamos o consumo energético. Consegues apontar algo nessa direção? Por qual motivos os modelos que treinamos acham que o código vai usar muito mais energia do que realmente usa? E o que podemos fazer para melhorar isso?}


\section{Evaluating Java Benchmark Programs} \label{sec:evaluating_java_benchmark_programs}

Our previous experiments were designed to evaluate simple methods, but we also used test cases considered more realistic, to further evaluate the framework’s ability to analyze and collect energy models for various custom programs. The programs were selected from the Computer Language Benchmarks Game~\cite{benchmarksGameJava}, which provides algorithmic benchmarks designed to test language performance on tasks that are representative of certain real-world computational workloads. The chosen programs were \texttt{BinaryTrees} (Java naot \#3) \texttt{fannkuch-redux} (Java), \texttt{n-body} (Java naot \#4), and \texttt{spectral-norm} (Java). These benchmarks were specifically selected because they are purely CPU and memory bound, with minimal variability, no reliance on disk I/O or multithreading, and no usage of external libraries. This makes them more predictable and reliable for energy consumption measurements.


{\color{blue}Starting with the program \texttt{BinaryTrees}, which implements a classic performance benchmark that builds, traverses, and destroys many binary trees to stress the memory allocation and garbage collection systems. It is composed of three main methods, some of which perform computationally intensive tasks, described as follows:}

\begin{itemize}
  \item \texttt{createTree:} Recursively builds a binary tree of the given depth. At each level, creates a node with two children (left and right), until depth equals zero. The result is a complete binary tree with \(2^{\text{depth}} - 1\) nodes.
 
  \item \texttt{checkTree:} Recursively traverses the tree and counts the number of nodes. For leaf nodes, it returns \texttt{1}. For internal nodes, it returns the sum of the recursive calls on the left and right subtrees plus one: $1 + \text{check(left)} + \text{check(right)}$.
  
  \item \texttt{trees:} Performs the core part of the benchmark by creating and checking many binary trees of varying depths and keeping one long-lived tree in memory to simulate memory pressure.
\end{itemize}

{\color{blue}The \texttt{fannkuch-redux} program also presents a computationally intensive task. It generates all permutations of the sequence \([0, 1, \ldots, n-1]\) and, for each permutation, performs a "flip" operation until the first element is zero. A flip consists of reversing the order of the first \(k + 1\) elements, where \(k\) is the first element of the permutation. For each permutation, it tracks the number of flips needed (called the flip count) and maintains the maximum flip count across all permutations. Additionally, it calculates a checksum by adding or subtracting the flip count based on the parity of the permutation index. The function returns the maximum number of flips needed for any permutation.

The \texttt{spectralnorm} program computes an approximation of the spectral norm (largest singular value) of a specific, implicitly defined matrix. It uses the power method, an algorithm that repeatedly multiplies a vector by the matrix to estimate how strongly the matrix can amplify a vector. The matrix \(A\) is not stored explicitly, instead, each element is computed on demand using the formula \(A_{i,j} = \frac{1}{\frac{(i + j)(i + j + 1)}{2} + i + 1}\). The program initializes a unit vector and performs 20 steps of power iteration by repeatedly multiplying with \(A\) and its transpose \(A^T\). The spectral norm is then approximated as \(\sqrt{\frac{v^T (A^T A) v}{v^T v}}\), where \(v\) is the resulting vector after the iterations. 

The final program is the \texttt{nbody} program, which simulates the motion of celestial bodies under the influence of gravity, modeling a simplified version of the solar system. It tracks five bodies: the Sun, Jupiter, Saturn, Uranus, and Neptune. The simulation starts by calculating the total momentum of the system and adjusting the Sun’s velocity to keep the center of mass stationary. Then, for a given number of time steps, it repeatedly updates the velocities and positions of the bodies using Newton’s law of universal gravitation. Each iteration involves computing pairwise gravitational interactions between all bodies, updating their velocities based on the forces, and then advancing their positions. The program prints the total system energy before and after the simulation to verify that it remains nearly constant, which serves as a correctness check. 
}


The Figure~\ref{fig:binarytrees_energy_panel_plot} \wo{Não gosto muito disso como três figuras separadas. Podem ser três plots na mesma figura ou as figuras lado a lado pelo menos. Parecido com o que tu fez antes com o MSE e R2} shows the energy of each method, for different input sizes, for every program generated and analyzed by the framework. It is possible to see that the methods have an exponential tendency with the increase of the input, which is expected from binary trees operations. It is also noticeable that the \texttt{checkTree} has a lower curve and maximum input, this was due to the fact that the method required that a tree was created before the method could be run, which would take more time, subsequently, the maximum input for this method was lower. 

\begin{figure}[htbp]
  \centering
  \includegraphics[width=\textwidth]{figures/binarytrees_energy_panel_plot.pdf}
  \caption{Energy for \texttt{BinaryTrees} methods}
  \label{fig:binarytrees_energy_panel_plot}
\end{figure}

\begin{comment}
  
\begin{figure}[htbp]
  \centering
  \includegraphics[width = .7 \textwidth]{figures/createTree_plot.pdf}
  \caption{Energy for method \texttt{createTree}}
  \label{fig:createTree_plot}
\end{figure}

\begin{figure}[htbp]
  \centering
  \includegraphics[width = .7 \textwidth]{figures/checkTree_plot.pdf}
  \caption{Energy for method \texttt{checkTree}}
  \label{fig:checkTree_plot}
\end{figure}

\begin{figure}[htbp]
  \centering
  \includegraphics[width = .7 \textwidth]{figures/trees_plot.pdf}
  \caption{Energy for method \texttt{trees}}
  \label{fig:trees_plot}
\end{figure}
\end{comment}

\begin{comment}
  \begin{table}[htbp]
  \centering
  \footnotesize
  \setlength{\tabcolsep}{10pt} % increase column separation
  \makebox[\textwidth][c]{%
    \begin{tabular}{@{}p{5.3cm}@{\hspace{2.5em}}c@{\hspace{1em}}c@{\hspace{1em}}c@{\hspace{1em}}c@{}}
      \toprule
      Method & Input Value & Predicted Energy (J) & Actual Energy Range (J) & Prediction Error (\%) \\
      \toprule

      \multirow{3}{*}{\texttt{BinaryTrees.createTree(int)}}
        & 5 & -0.082 & 1.27e-5 -- 1.49e-5 & 598070 \\
        & 10 & -0.083 & 3.75e-4 -- 4.60e-4 & 19980 \\
        & 23 & 10.26 & 11.78 -- 12.63 & 15.9 \\
        & 26 & 79.90 & 77.29 -- 80.63 & 1.2 \\
      \midrule

      \multirow{3}{*}{\texttt{BinaryTrees.checkTree(TreeNode)}}
        & 5 & 2.71e-05 & 2.63e-6 -- 2.70e-6 & 916 \\
        & 10 & 1.88e-4 & 1.64e-4 -- 1.75e-4 & 10.9 \\
        & 23 & 2.19e-3 & 1.37 -- 1.56 & 99.85 \\
        & 26 & 3.15e-3 & 11.72 -- 13.29 & 99.97 \\
      \midrule

      \multirow{3}{*}{\texttt{BinaryTrees.trees(int)}}
        & 5 & -9.94 & 2.68e-4 -- 3.55e-4 & 1265924 \\
        & 10 & -3.42 & 0.041 -- 0.045 & 8053 \\
        & 23 & 517.89 & 713 -- 800 & 31.59 \\
        & 26 & 2429 & 6260 -- 7111 & 63.66 \\
      \midrule

      \multirow{3}{*}{\shortstack[l]{\texttt{BinaryTrees.checkTree(TreeNode)} +\\\texttt{createTree(int)} +\\\texttt{trees(int)}}}
        & 5 & -4.03 & 2.77e-4 -- 3.44e-4 & 1296569 \\
        & 10 & -3.49 & 4.59e-2 -- 4.77e-2 & 7578 \\
        & 23 & 528.15 & 796 -- 806 & 34.08 \\
        & 26 & 2509 & 7062 -- 7350 & 65.18 \\
      \bottomrule
    \end{tabular}%
  }
  \caption{Comparison of actual and predicted energy consumption for BinaryTrees program}
  \label{tab:energy_comparison_bin_trees}
\end{table}
\end{comment}




%\begin{table}[htbp]
%  \centering
%  \footnotesize
%  \setlength{\tabcolsep}{10pt}
%  \makebox[\textwidth][c]{%
%    \begin{tabular}{@{}p{5.3cm}@{\hspace{2em}}c@{\hspace{1em}}c@{\hspace{1em}}c@{\hspace{1em}}c@{}}
%      \toprule
%      Method & Input Value (\%) & Predicted Energy Simple Model (J) & Predicted Energy Best Model (J) & Actual Energy Range (J) & Prediction Error Simple Model (\%) & Prediction Error Best Model (\%) \\
%      \toprule
%
%      \multirow{3}{=}{\texttt{BinaryTrees.createTree(int)}} 
%        & 10 & -0.083 & 5.63e-04 & 3.75e-4 -- 4.60e-4 & 19980 & 34.97 \\
%        & 23 & 10.26 & 9.42 & 11.78 -- 12.63 & 15.9 & 22.83 \\
%        & 26 & 79.90 & 80.11 & 77.29 -- 80.63 & 1.2 & 1.45 \\
%      \midrule
%
%      \multirow{3}{=}{\texttt{BinaryTrees.checkTree(TreeNode)}}
%        & 10 & 1.88e-4 & 2.28e-04 & 1.64e-4 -- 1.75e-4 & 10.9 & 34.4 \\
%        & 23 & 2.19e-3 & 2.43e-03 & 1.37 -- 1.56 & 99.85  & 99.83\\
%        & 26 & 3.15e-3 & 3.53e-03 & 11.72 -- 13.29 & 99.97 & 99.97\\
%      \midrule
%
%      \multirow{3}{=}{\texttt{BinaryTrees.trees(int)}}
%        & 10 & -3.42 & 1.69 & 0.041 -- 0.045 & 8053 & 3833 \\
%        & 23 & 517.89 & 798 & 713 -- 800 & 31.59 & 5.49\\
%        & 26 & 2429 & 39754 & 6260 -- 7111 & 63.66 & 494.63 \\
%      \midrule
%
%      \multirow{3}{=}{\parbox{5.3cm}{\centering\texttt{BinaryTrees.checkTree(TreeNode) +\\ createTree(int) + trees(int)}}}
%        & 10 & -3.49 & 4.59e-2 -- 4.77e-2 & 7578 \\
%        & 23 & 528.15 & 796 -- 806 & 34.08 \\
%        & 26 & 2509 & 7062 -- 7350 & 65.18 \\
%      \bottomrule
%    \end{tabular}%
%  }
%  \caption{Comparison of actual and predicted energy consumption for BinaryTrees program}
%  \label{tab:energy_comparison_bin_trees}
%\end{table}

Table~\ref{tab:energy_comparison_bin_trees} presents the detailed information about the energy consumption of the BinaryTrees program\wo{de novo os resultados parecem super estimados. Tens uma intuição sobre isso?}. It is noticeable that the values predicted are not the most accurate, for some inputs it has good accuracy, for example, the method \texttt{createTree} has good accuracy for the input \texttt{26}. For other methods, such as \texttt{trees} it does not. Again as the previous experiment, the tool is not very accurate in terms of absolute values, and in this case for methods that change their energy considerably with minimum increase in the input, the models have a harder time to predict. The tool demonstrates good relative energy prediction, as input values increase, the predicted energy consumption also increases, and it decreases correspondingly when the inputs are reduced.


\begin{table}[htbp]
  \centering
  \footnotesize
  \setlength{\tabcolsep}{10pt} % increase column separation
  \makebox[\textwidth][c]{%
    \begin{tabular}{@{}p{5.3cm}@{\hspace{2.5em}}c@{\hspace{1em}}c@{\hspace{1em}}c@{\hspace{1em}}c@{}}
      \toprule
      Method & Input Value & Predicted Energy (J) & Actual Energy Range (J) & Prediction Error (\%) \\
      \toprule

      \multirow{3}{*}{\texttt{BinaryTrees.createTree(int)}}
        & 5 & -0.082 & [1.27e-5, 1.49e-5] & 598070 \\
        & 10 & -0.083 & [3.75e-4, 4.60e-4] & 19980 \\
        & 23 & 10.26 & [11.78, 12.63] & 15.9 \\
        & 26 & 79.84 & [77.29, 80.63] & 1.12 \\
      \midrule

      \multirow{3}{*}{\texttt{BinaryTrees.checkTree(TreeNode)}}
        & 5 & 2.71e-05 & [2.63e-6, 2.70e-6] & 916 \\
        & 10 & 1.88e-4 & [1.64e-4, 1.75e-4] & 10.9 \\
        & 23 & 2.19e-3 & [1.37, 1.56] & 99.85 \\
        & 26 & 3.15e-3 & [11.72, 13.29] & 99.97 \\
      \midrule

      \multirow{3}{*}{\texttt{BinaryTrees.trees(int)}}
        & 5 & 0.02 & [2.68e-4, 3.55e-4] & 7202 \\
        & 10 & 0.38 & [0.041, 0.045] & 786 \\
        & 23 & 581 & [713, 800] & 23.23 \\
        & 26 & 3152 & [6260, 7111] & 52.86 \\
      \midrule

      \multirow{3}{*}{\shortstack[l]{\texttt{BinaryTrees.checkTree(TreeNode)} +\\\texttt{createTree(int)} +\\\texttt{trees(int)}}}
        & 5 & -0.06 & [2.77e-4, 3.44e-4] & 19341 \\
        & 10 & 0.30 & [4.59e-2, 4.77e-2] & 541 \\
        & 23 & 591 & [796, 806] & 26.20 \\
        & 26 & 3232 & [7062, 7350] & 55 \\
      \bottomrule
    \end{tabular}%
  }
  \caption{Comparison of actual and predicted energy consumption for BinaryTrees program}
  \label{tab:energy_comparison_bin_trees}
\end{table}

\wo{acho que podes juntar as tabelas 5.4, 5.5 e 5.6 em uma só. Pq só falas dos métodos de BinaryTrees, e não dos outros? As figuras falam nomes de métodos sem dizer de onde vem. Os benchmarks não tem metodos? Eu não gosto muito de todas essas tabelas e plots separados. Vê se consegues juntar mais.}


The Figure~\ref{fig:benchmark_energy_panel_plot}, shows the measured energy with the predictions, and it is observable that the energy grows quickly as the input also increases. Both the Table~\ref{tab:energy_comparison_benchmark} and the Figure~\ref{fig:benchmark_energy_panel_plot}, confirm that for inputs lower than ten, the model has difficulties predicting.


{\color{blue}For the \texttt{spectralnorm} benchmark, several methods are involved: \texttt{Approximate(int)}, \texttt{A(int, int)}, \texttt{MultiplyAv(int, \allowbreak\ double[],\allowbreak\ double[])}, \texttt{MultiplyAtv(int,\allowbreak\ double[],\allowbreak\ double[])}, and \texttt{MultiplyAtAv(int,\allowbreak\ double[],\allowbreak\ double[])}. While all these methods were included during the training, the analysis focused on \texttt{Approximate(int)}. This is because it is the method invoked by the \texttt{main} function and is responsible for the core computation of the program.}
The model obtained for the \texttt{spectralnorm} case prove quite effective discovering the pattern of the energy usage of the method, being the input the main feature that help achieve it, as Figure~\ref{fig:benchmark_energy_panel_plot} shows, the prediction and measured values are almost similiar, and the prediction follows the curve of the mesaured energy. The concrete values can be seen in the Table~\ref{tab:energy_comparison_benchmark}, and it illustrates that the higher inputs tend to have higher accuracy as well.


{\color{blue}For the \texttt{spectralnorm} benchmark, several methods are involved: \texttt{offsetMomentum(\allowbreak\ double,\allowbreak\ double,\allowbreak\ double)},
\texttt{energy()}, \texttt{advance(double)}. The methods included in the training were \texttt{energy()}, \texttt{advance(double)}. However, the \texttt{energy()} method did not receive any inputs and lacked features that the model could use to differentiate energy consumption across executions. As a result, the model learned to predict a constant value for this method. Consequently, only \texttt{advance(double)} was selected for detailed analysis in this benchmark experiment.}
This time the resulting model had difficulties predicting the energy for the given method, as shown in the Table~\ref{tab:energy_comparison_benchmark}. The Figure~\ref{fig:benchmark_energy_panel_plot} also shows an interesting behavior of the measured energy, which might explain why the model accuracy is so low.


\begin{table}[htbp]
  \centering
  \footnotesize
  \setlength{\tabcolsep}{10pt} 
  \makebox[\textwidth][c]{%
    \begin{tabular}{@{}p{5.3cm}@{\hspace{2.5em}}c@{\hspace{1em}}c@{\hspace{1em}}c@{\hspace{1em}}c@{}}
      \toprule
      Method & Input Value & Predicted Energy (J) & Actual Energy Range (J) & Prediction Error (\%) \\
      \toprule

      \multirow{3}{*}{\texttt{NBodySystem.advance(double)}}
        & 100 & 4.4e-5 & [3.02e-6, 3.25e-6] & 1302 \\
        & 1000 & 4.85e-5 & [2.93e-6, 3.20e-6] & 1480 \\
        & 10,000 & 4.11e-5 & [2.97e-6, 3.15e-6] & 1243 \\
        & 100,000 & 5e-5 & [2.97e-6, 3.15e-6] & 1533 \\
      \midrule
      \multirow{3}{*}{\texttt{fannkuch(int)}}
        & 5 & 3.68e-3 & [8.04e-5, 9.51e-5] & 4089 \\
        & 10 & 8.69 & [3.72, 8.52] & 42 \\
        & 11 & 105.26 & [87.25, 111.36] & 6 \\
        & 12 & 1275 & [1187, 1585] & 8 \\
      \midrule
      \multirow{3}{*}{\texttt{spectralnorm().Approximate(int)}}
        & 10 & 6.29e-3 & [2.65e-4, 2.85e-4] & 2187 \\
        & 100 & .08 & [0.02, 0.021] & 297 \\
        & 1000 & 2.67 & [2.0, 2.22] & 26 \\
        & 10,000 & 212 & [205, 207] & 2.9 \\
      \bottomrule
    \end{tabular}%
  }
  \caption{Comparison of actual and predicted energy consumption for nBody, fannkuch and spectralnorm program}
  \label{tab:energy_comparison_benchmark}
\end{table}


\begin{comment}
  \begin{table}[htbp]
  \centering
  \footnotesize
  \setlength{\tabcolsep}{10pt}
  \makebox[\textwidth][c]{%
    \begin{tabular}{@{}p{5.3cm}@{\hspace{2.5em}}c@{\hspace{1em}}c@{\hspace{1em}}c@{\hspace{1em}}c@{}}
      \toprule
      Method & Input Value & Predicted Energy (J) & Actual Energy Range (J) & Prediction Error (\%) \\
      \toprule

      \multirow{3}{*}{\texttt{fannkuch(int)}}
        & 5 & 3.68e-3 & [8.04e-5, 9.51e-5] & 4089 \\
        & 10 & 8.69 & [3.72, 8.52] & 42 \\
        & 11 & 105.26 & [87.25, 111.36] & 6 \\
        & 12 & 1275 & [1187, 1585] & 8 \\
      \bottomrule
    \end{tabular}%
  }
  \caption{Comparison of actual and predicted energy consumption for fannkuch redux program}
  \label{tab:energy_comparison_fannkuch_redux}
\end{table}

\begin{table}[htbp]
  \centering
  \footnotesize
  \setlength{\tabcolsep}{10pt}
  \makebox[\textwidth][c]{%
    \begin{tabular}{@{}p{5.3cm}@{\hspace{2.5em}}c@{\hspace{1em}}c@{\hspace{1em}}c@{\hspace{1em}}c@{}}
      \toprule
      Method & Input Value & Predicted Energy (J) & Actual Energy Range (J) & Prediction Error (\%) \\
      \toprule

      \multirow{3}{*}{\texttt{spectralnorm().Approximate(int)}}
        & 10 & 6.29e-3 & [2.65e-4, 2.85e-4] & 2187 \\
        & 100 & .08 & [0.02, 0.021] & 297 \\
        & 1000 & 2.67 & [2.0, 2.22] & 26 \\
        & 10,000 & 212 & [205, 207] & 2.9 \\
      \bottomrule
    \end{tabular}%
  }
  \caption{Comparison of actual and predicted energy consumption for spectral-norm program}
  \label{tab:energy_comparison_spectral_norm}
\end{table}
\end{comment}



\begin{figure}[htbp]
  \centering
  \includegraphics[width=\textwidth]{figures/benchmark_energy_panel_plot.pdf}
  \caption{Energy for \texttt{advance}, \texttt{Approximate} and \texttt{fannkuch} methods}
  \label{fig:benchmark_energy_panel_plot}
\end{figure}



\begin{comment}
  

\begin{figure}[htbp]
  \centering
  \includegraphics[width = .7 \textwidth]{figures/advance_plot.pdf}
  \caption{Energy for method \texttt{advance}}
  \label{fig:advance_plot}
\end{figure}

\begin{figure}[htbp]
  \centering
  \includegraphics[width = .7 \textwidth]{figures/approximate_plot.pdf}
  \caption{Energy for method \texttt{approximate}}
  \label{fig:approximate_plot}
\end{figure}

\begin{figure}[htbp]
  \centering
  \includegraphics[width = .7 \textwidth]{figures/fannkuch_plot.pdf}
  \caption{Energy for method \texttt{fannkuch}}
  \label{fig:fannkuch_plot}
\end{figure}
\end{comment}


To achieve better results it could be possible to improve the feature's selection by incorporating characteristics that better capture the program complexity and increase the maximum input available for each program generated in order to better represent the high-energy regions of execution.


This experiment also tests how pratical it is to actually measure a custom program. 
The \texttt{BinaryTrees} program, from how it is implemented on the website it needs some small adjustments in order to work with our framework. First, the program does not have a constructor for the class \texttt{TreeNode} which does not allow the program generator to populate the programs correctly, as the program on the website uses a custom method that create the nodes. Without this change, the methods that required \texttt{TreeNode} as input would be simply receiving an empty Object, as the constructor was also empty, making their generation useless as every program would be the same.
With that said, the only adjustments made to the class were, a new private empty constructor, that can only be accessed inside the class, so the generator does not detect it and use it, as the generator always tries to use the smallest one available, so it is important that the empty constructor is private. And it was added a new constructor, this one public, that had the same logic as the method that created the nodes. Now the methods that use \texttt{TreeNode} can be generated as the constructor can call the public constructor that has the similar behavior of the \texttt{createTree(int)} method. These changes do not impact how the program, or methods behave and only helps the program generation.
Aside from the \texttt{BinaryTrees} program, the others were even less troubled, as the \texttt{fannkuch-redux} and \texttt{spectral-norm} required no changes, apart from the package declaration, and the \texttt{n-body} program only required to divide the internal classes into more Java classes resulting in three files — \texttt{n-body.java}, \texttt{NBodySystem.java}, and \texttt{Body.java}.


This experiment objective's was not focused only on achieving the highest accuracy possible. It was also made to prove that it is possible to use the framework to test more interesting and complex cases. And it was proven that with minimal changes, it was possible to mass generate programs, collect data and train models, for four different programs. While it may not work universally for all programs, since creating a framework general enough to handle every case is challenging, this experiment demonstrates that analyzing custom programs is possible.


\section{Analysis} \label{sec:analysis}

In this chapter the results obtain for the custom programs will be analyzed.

In most of the experiments, it was observed that the energy consumption curve followed the input values, meaning the model's predicted energy usage was largely dependent on the input. Since the tested programs involved computationally intensive tasks, increasing the input also led to longer execution times, which in turn caused higher energy consumption, indicating a strong correlation between execution time and energy usage. 
For the methods where a lower input was enough to generate a benchmark that would run for an acceptable amount of time, the models could predict better, as a small increase in the input would affect the energy consumption by a significant amount. However, to prioritize accurate predictions for higher input values, many models tended to fail in predicting energy usage for the lower input range.

As it is now, the most important feature collected from the benchmarks is the input, and it showed to be effective in helping the models outputting a reliable prediction model. However, it is possible to see that it is the only variable that can help the models predict in this case. They could benefit from a more diverse set of features that could include dynamic features, for instance, time of execution, processor cycles or other low-level metrics. Using more features could help the overall accuracy of the models.

From the experiments conducted, the one that displayed the most unusual result, was the \texttt{advance} method from the \texttt{nbody} program. The concrete reason to why the predictions and the measured values are not as accurate as the those of other methods, is not known for sure, however, there is a hypothesis. The methods \texttt{advance} required higher inputs. Since the input generator is limited to a maximum threshold, the benchmarks generated were also capped to that value. This would make that an input of 1 or 100,000 would spend in general the same amount of energy, making the input a bad decision factor for the model. Increasing the input could lead to a better model for this particular method, but could also increase the search time of the maximum inputs in the generator.

Overall, the results indicate that while the models are capable of producing reliable energy consumption predictions under certain conditions, their accuracy and adaptability could be significantly improved with a richer and more diverse feature set.



%\wo{Escreve aqui uma secção de análise, onde tu analisas os resultados que obtiveste. O que é que tu achas que correu bem? O que é que correu mal? Pq o resultados do Nbody são tão ruins? pq o fannkuch é incrivelmente bom? Pq a maioria deles consegue prever melhor quando temos um input maior? Essa é parte que vai mostrar que tu percebeu o que se passou no teu trabalho.}






\section{Limitations and Challenges} \label{sec:limitations_and_challenges}

During program generation, most of the tested collections focused on \texttt{List}, \texttt{Set}, and \texttt{Map}. However, programs were also generated for another collection category, the \texttt{Math} library. This introduced an issue. The computations in these programs were extremely fast, completing too quickly for PowerJoular to measure any meaningful energy consumption. As a result, all energy readings for the \texttt{Math} programs were reported as 0J.
We were able to identify that the source of the problem was the array sizes used to hold the method parameters. While the three predefined array sizes worked well for \texttt{List}, \texttt{Set}, and \texttt{Map}, they were too small for the \texttt{Math} library. The smaller input sizes caused the Math computations to execute very fast, not giving PowerJoular enough time to capture energy usage.
This highlights a challenge in energy profiling: some collections, especially those involving lightweight or highly optimized operations like Math functions, may be difficult to measure accurately. One possible solution would be to determine array sizes during program generation, large enough to ensure that each program runs for at least one second, giving PowerJoular sufficient time to measure energy consumption. However, implementing this would significantly increase the time required for program generation, as it would involve exploring many more combinations of input sizes to find suitable configurations. \wo{A gente já não faz isso?}

Another important factor is that the tool is not able to predict energy when threads are involved, as the programs generated only used a single thread, subsequently the models will not have that factor into account. This limitation exists because measuring and modeling the energy usage of multithreaded programs is particularly challenging due to factors like concurrent execution, thread scheduling, and synchronization overhead. However, threading can deeply impact the energy usage of a program, as a program execution stop being strictly linear, and can have multiple simultaneous computations, potentially increasing the code energy consumption.


A limitation detected during the experimentation of analyzing an external program, was how the inputs on some programs are defined. For example, a program can have a method that receives a \texttt{String}, however, when inside the method it decides to convert the \texttt{String} to \texttt{int} and use it as an \texttt{Integer}, making it so a method that actually uses \texttt{Integers} and not \texttt{Strings}. Although it is easily noticeable by human eye, the program generator cannot understand this kind of context changes as easy, and it would start generating inputs for \texttt{Strings} instead of \texttt{Integers}, which would fail in the program generation. A good example of this usage is when the main function is called and receives the arguments in the \texttt{String[]} and then it uses its values in different variables.





%In the start of the project, some tools were tested in order to see how to get the energy profiles for later use. The tools tested were PowerJoular, Powertop, Perf and JoularJx.
%
%Perf is a Linux tool primarily designed for analyzing application performance characteristics rather than precise energy measurement. While it can provide some energy-related metrics, its measurements tend to be imprecise. In this context, Perf was used mainly to get a rough idea of energy consumption and to serve as an alternative when more accurate tools were unavailable.
%
%Powertop was also tested, but it could only perform energy measurements on laptops, as it relies on battery drain data to calculate energy consumption. Since this approach doesn't align with our specific requirements, we considered Powertop as a last-resort option.
%
%JoularJx is an energy measurement tool capable of measuring the consumption of Java programs and its methods. However, it is not as precise as other tools as it requires to measure the entire start of the JVM and whole functions instead of small code blocks.
%
%As described in the \ref{sec:background_energy}, PowerJoular is the best option. As a command line program it can be easily adapted to measure any program or code snippet in most languages. So it was combined with the framework experiment-runner, that facilitates the process.
%
%Initially, JoularJx and the experiment-runner using PowerJoular were used to explore their capabilities and familiarize with the tools. A sample Fibonacci program written in Java and C was used as a test case. However, the energy measurements provided by the two tools differed, and the experiment runner occasionally encountered errors. Later, it was determined that these problems were due to incorrect use of the framework. However, it was still decided the best approach was to make a new orchestrator similar to the Experiment-Runner, but simpler and specifically focused on measuring process energy consumption using PowerJoular. A Java-based orchestrator was initially developed because the Fibonacci implementation was also in Java. However, when tested, the energy measurement results differed significantly from those of the experiment runner, even though the main difference was the programming language (Java vs. Python).
%
%To further analyze these inconsistencies, another orchestrator was developed in Python. This allowed for a closer examination of the differences in energy measurements and a deeper understanding of the behavior of the tools.
%
%Using the process explained on \ref{chapter:approach} it was noticeable that the Java orchestrator was getting significantly more energy consumption than the Python one, which is not very logical, since they both target the same program. So, to try and check which one was having problems, two more orchestrators were implemented, one in C and another in bash.
%
%After running the tests again it was possible to see that the Python orchestrator was getting values way more different from the other three orchestrators as show in Figure \ref{fig:4_orchs_comparison}.
%The figure contains 100 runs of the Fibonacci recursive program written in Java and order by the less energy to the highest energy. And it shows the energy reads for the four different orchestrators used. The labels contain the average energy values and its standard deviation.
%
%Further analysis of the orchestrators revealed a notable difference in behavior. When the Python orchestrator was running, both the parent and child processes consumed CPU resources. In contrast, the other orchestrators (Java, C, and Bash) showed CPU usage only in the child process. This disparity may explain why PowerJoular reported lower energy consumption for the Python orchestrator. Since the CPU load was shared between the parent and child processes, PowerJoular, which measures energy only for the child process (the target Fibonacci program), captured less total energy usage.
%Since the experiment runner included an example demonstrating how to use the framework with PowerJoular, the authors were made aware of this potential conflict when launching PowerJoular from Python.
%
%\begin{figure}%[h]
%  \centering
%  \includegraphics[width = 0.5 \textwidth]{figures/4_orchestrators_comparison.pdf}
%  \caption{orchestrators comparison}
%  \label{fig:4_orchs_comparison}
%\end{figure}

% Conclusao
% Conclusao

\chapter{Conclusion}\label{chapter:conclusion}


The techniques and tools in past and recent research have been studied, and this work improves on those different research and tools, by providing a framework that can be used to obtain models that can estimate energy consumption, this thesis combines the usage of machine learning with static analysis for energy estimation.
The framework presented is divided in 4 modules, and it eases the process of obtaining the estimation models, by generating the necessary data, collecting the energy profiles, training the dataset and presenting the results. By simply providing a program or selecting a Java collection, the framework can generate estimation models adapted to the input. 

For developers who prefer not to wait for the full analysis process or are less familiar with energy-aware programming, our VSCode extension offers a quick and accessible alternative. By using the pre-trained models the tool provides immediate feedback on the estimated energy consumption of their code, without the need to execute it, helping raise awareness of energy efficiency in software development. 

During the experiments, the framework proved it was capable of evaluating custom programs, with minimal modifications, which makes it reliable to obtain more data from across a wider variety of codebases. Additionally, the extension tool also proved it can raise energy awareness by using the trained models and static analysis, to provide real-time feedback on the energy impact of code changes.

In summary, this thesis contributes to the field of energy-aware programming by combining machine learning and static analysis to enable accurate and accessible energy prediction. Nonetheless, continued research and development are essential to further improve model accuracy, expand feature coverage, and support a broader range of programming scenarios.


%future work
\chapter{Future Work}

Building on the progress made so far, additional energy profiles must be created to develop a robust energy inference function. This task involves collecting profiles from different machines and gathering enough data to effectively train the machine learning model. The static analysis tool will be implemented and used to provide important code, specifically, certain features that will feed into the model.

After that the efforts will be on developing the actual energy inference function. This task is expected to be the most time-consuming, as it is crucial to the tool functioning. It is most likely that this task overlaps the energy profiling task as more profiles might need to be collected. The task will start by implementing the ML algorithms described in \ref{sec:background_machine_learning}, test their accuracy and improving it by tuning the energy profiles for many possible cases.

When the function is completed, it will be tested against other tools to ensure its quality. The extension will then be built and tested in IDEs.
While developing the function and the extension, the writing of the thesis will also be done in parallel documenting every information.

\begin{figure*}%[h]
  \centering
  \includegraphics[width = 1 \textwidth]{figures/gantt_diagram.pdf}
  \caption{Work Plan}
  \label{fig:gantt_diagram}
\end{figure*}


The work plan is illustrated more clearly in Figure \ref{fig:gantt_diagram}, which provides a visual representation of the described tasks. While the dates shown in the figure may not precisely align with the actual timeline, the sequence of tasks remains accurate.

% Fim do conteudo
% ----------------------------------------------------------------------

% Glossario

%\LIMPA

%
% Para actualizar o glossario, e' preciso correr o script ./fazindice
% e voltar a gerar o PDF
%
\include{./tex/acronimos}
%\renewcommand{\glossaryname}{Abreviaturas}
%\setglossary{glodelim={\noexpand}}
%\setglossary{glsnumformat=ignore}
\printglossary
\addcontentsline {toc} {chapter} {\glossaryname}

% Bibliografia

%\LIMPA
\nocite{*}
\bibliographystyle{unsrt}
\bibliography{bibliography}
\addcontentsline {toc} {chapter} {\bibname}

% Indice remissivo
%\LIMPA
\printindex
\addcontentsline {toc} {chapter} {\indexname}

% Inicio apendices
\appendix

\include{./tex/apendices}
\end{document}
