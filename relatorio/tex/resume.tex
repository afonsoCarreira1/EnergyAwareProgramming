\selectlanguage{portuguese}
\vspace*{2cm}
\begin{center}
\Large \bf Resumo
\end{center}
\vspace*{1cm} \setlength{\baselineskip}{0.6cm}

Nos últimos anos, o uso e a gestão da energia tornaram-se uma questão global. Está em curso a procura de energias renováveis que reduzam o impacto ecológico no nosso planeta. No entanto, estas alternativas não são tão acessíveis nem tão consistentes como as opções tradicionais. Existem muitas áreas em que é possível reduzir a pegada energética, e o sector da informação e tecnologia é um deles.
Poupar energia ao nível de aplicações é importante para todas as categorias de dispositivos eletrónicos desde dispositivos móveis a centros de dados. {\color{blue}No entanto, estima-se que, em 2020, cerca de 7\% da energia elétrica global tenha sido utilizada para fins de tecnologia da informação e comunicação. Com a crescente procura por novas tecnologias, prevê-se que esse número continue a aumentar nos próximos anos. }
Esta tendência mostra que tem que existir um cuidado maior por parte dos programadores na fase de desenvolvimento de código que seja energeticamente eficiente. No entanto, quando de facto tentam procurar formas de reduzir os custos energéticos dos seus programas, e recorrem a \textit{websites} ou grandes modelos linguísticos, e muitas vezes as respostas obtidas podem não estar completamente corretas. Existe ainda uma necessidade de ferramentas de fácil utilização que permita que os programadores rapidamente e sem grande conhecimento na área da energia consigam perceber a energia utilizada pelos seus programas.
Esta tese tem como objectivo aumentar a consciencialização dos programadores para a energia dos seus programas, apresentando uma \textit{framework} que facilita a obtenção de modelos que realizam a previsão de energia. Ela segue um fluxo estruturado que começa com a geração de vários programas usando várias categorias de coleções ou classes definidas pelo programador. Esses programas são então submetidos a um análise que permite obter perfis energéticos para treinar modelos para cada método na classe alvo.
Os modelos resultantes são integrados numa extensão de ambiente de desenvolvimento integrado capaz de identificar o consumo de energia dos métodos nos programas e apresentar essas informações rapidamente aos programadores, permitindo-lhes tomar decisões informadas no desenvolvimento de aplicações.
Essa maior consciencialização permitirá que os programadores compreendam o impacto geral das suas escolhas de programação no consumo e na eficiência energética.
Para ser possível obter os modelos que consigam prever a energia de programas, é primeiro necessário ter uma quantidade considerável de dados. Para isso, foi desenvolvido um gerador de programas~\wo{talvez seja tarde para isso mas acho que talvez seja bom mudar de "gerador de programas" para "gerador de benchmarks"}, com o objetivo de produzir automaticamente um grande número de exemplos com variações relevantes.
Este gerador permite criar programas para várias coleções de linguagem Java ou para programas personalizados. A geração é feita com base na coleção e no método especificado, produzindo múltiplas combinações através da variação dos tipos de variáveis e dos valores de entrada. Cada programa gerado foi estruturado para analisar exclusivamente um método específico, estando otimizado para a sua avaliação através de testes de desempenho.
A segunda etapa consiste em utilizar todos os programas gerados para analisar individualmente cada um, com o objetivo de obter os perfis de consumo de energia associados a cada método em estudo. Esta etapa consistiu em construir um processo eficiente de recolha dos dados de consumo energético dos programas, minimizando leituras desnecessárias e ruído, tendo em conta as limitações inerentes à medição de energia em aplicações. Para a recolha dos dados, foi utilizada a ferramenta PowerJoular, que permite monitorizar o consumo energético de forma precisa durante a execução dos programas.
Para ilustrar o funcionamento da framework neste trabalho foram recolhidos dados para as coleções \texttt{List} e \texttt{Map} da linguagem Java, que serviram de base para o treino e validação dos modelos.
Após a obtenção dos perfis de energia, foi possível treinar modelos de aprendizagem automática supervisionada com os dados recolhidos. Nesta fase, foram comparados diversos algoritmos, tendo-se optado pelo PySR, uma ferramenta que gera um conjunto de expressões matemáticas ordenadas da mais simples para a mais complexa. Esta abordagem permite equilibrar entre desempenho e interpretabilidade, oferecendo modelos transparentes, ao contrário dos modelos de caixa preta, e facilitando a visualização da expressão utilizada para prever o consumo energético.
Por fim foi construida uma extensão para o VSCode que integra os modelos previamente treinados. Esta ferramenta combina análise estática com aprendizagem automática para realizar previsões de energia de programas. Esta combinação permite que o cálculo da energia utilizada seja bastante rápido, evitando a necessidade de executar programas, e aumentando a consciencialização dos programadores para a eficiência energética dos seus programas.
Os resultados obtidos mostram que a framework consegue gerar modelos para programas personalizados necessitando apenas de alterações mínimas. A ferramenta de previsão de energia revelou-se útil para identificar variações relativas no consumo energético, à medida que os valores de entrada aumentam, observa-se um aumento correspondente nos valores de energia. No entanto, ao nível de valores absolutos, a precisão das previsões depende diretamente do desempenho dos modelos treinados, sendo necessária uma melhoria nestes para obter resultados mais precisos.

Em resumo, esta tese contribui para o campo da programação com consciência energética, combinando aprendizagem automática e análise estática para permitir uma previsão energética precisa e acessível. No entanto, a investigação e o desenvolvimento contínuos são essenciais para melhorar ainda mais a precisão do modelo, expandir a cobertura de funcionalidades e apoiar uma variedade mais ampla de cenários de programação.


\vfill

\begin{flushleft}
\textbf{Keywords:}
Computação ecológica, Análise estática de código, Previsão de energia, Software sustentável, Programação com consciência energética
\end{flushleft}

%\LIMPA
%\selectlanguage{portuguese}