\selectlanguage{portuguese}
\vspace*{2cm}
\begin{center}
\Large \bf Resumo
\end{center}
\vspace*{1cm} \setlength{\baselineskip}{0.6cm}

Nos últimos anos, o uso e a gestão da energia tornaram-se uma questão global. Está em curso um esforço global para identificar energias renováveis que reduzam o impacto ecológico no nosso planeta. No entanto, estas alternativas não são tão acessíveis nem tão consistentes como as opções tradicionais, o que limita a sua utilização em muitos contextos. Existem diversas áreas onde é possível implementar medidas para reduzir significativamente a pegada energética, sendo o sector da informação e tecnologia um dos mais relevantes e com grande potencial de impacto.
Poupar energia ao nível de aplicações é importante para todas as categorias de dispositivos eletrónicos desde dispositivos móveis a centros de dados. No entanto, já em 2020 estimava-se que cerca de 7\% da energia elétrica global tenha sido utilizada para fins de tecnologia da informação e comunicação. Com o aumento contínuo da procura por novas tecnologias e pela constante inovação no sector, prevê-se que esse número seja, atualmente, significativamente mais elevado do que em anos anteriores, e que continue a crescer de modo contínuo ao longo dos próximos anos.

Esta tendência mostra que tem que existir um cuidado redobrado por parte dos programadores na fase de desenvolvimento, de forma a garantir que o código seja energeticamente eficiente. No entanto, quando de facto tentam procurar formas de reduzir os custos energéticos dos seus programas, e recorrem a \textit{websites} ou grandes modelos linguísticos, muitas vezes as respostas obtidas podem não estar completamente corretas. Existe ainda uma necessidade de ferramentas de fácil utilização que permita que os programadores rapidamente e sem grande conhecimento na área da energia consigam perceber a energia utilizada pelos seus programas.
Esta tese tem como objectivo aumentar a consciencialização dos programadores para a energia dos seus programas, apresentando uma \textit{framework} que facilita a obtenção de modelos que realizam a previsão de energia. Ela segue um fluxo estruturado que começa com a geração de vários \textit{benchmarks} usando várias categorias de coleções ou classes definidas pelo programador. Esses \textit{benchmarks} são então submetidos a um análise que permite obter perfis energéticos para treinar modelos para cada método na classe alvo.
Os modelos resultantes são integrados numa extensão de ambiente de desenvolvimento integrado capaz de identificar o consumo de energia dos métodos nos programas e apresentar essas informações rapidamente aos programadores, permitindo-lhes tomar decisões informadas no desenvolvimento de aplicações.
Essa maior consciencialização permitirá que os programadores compreendam o impacto geral das suas escolhas de programação no consumo e na eficiência energética, podendo levar a uma redução da energia consumida por sistemas informáticos.
Para ser possível obter os modelos que consigam prever a energia de programas, é primeiro necessário ter uma quantidade considerável de dados. Para isso, foi desenvolvido um gerador de \textit{benchmarks}, com o objetivo de produzir automaticamente um grande número de exemplos com variações relevantes.
Este gerador permite criar \textit{benchmarks} para várias coleções de linguagem Java ou para programas personalizados. A geração é feita com base na coleção e no método especificado, produzindo múltiplas combinações através da variação dos tipos de variáveis e dos valores de entrada. 
Cada método gera inicialmente um modelo que serve de base para a criação massiva dos \textit{benchmarks}. Este modelo possui a estrutura necessária para ser interpretada pelo orquestrador e contém vários marcadores de posição nas secções correspondentes aos tipos e valores de entrada, que serão substituídos posteriormente pelo gerador de \textit{benchmarks} com os valores concretos. A estrutura do modelo segue uma ordem específica, pensada para otimizar a eficiência na leitura do consumo energético de cada \textit{benchmark}, que é gerado de forma a analisar exclusivamente um método específico, estando otimizado para a sua avaliação através de testes de desempenho.

A segunda etapa consiste em utilizar todos os \textit{benchmarks} gerados para analisar individualmente cada um, com o objetivo de obter os perfis de consumo de energia associados a cada método em estudo.
Para a recolha dos dados, foi utilizada a ferramenta PowerJoular, que permite monitorizar o consumo energético de forma precisa durante a execução dos programas.
Esta etapa consistiu em construir um processo eficiente de recolha dos dados de consumo energético dos \textit{benchmarks}. O processo inicia-se com o orquestrador a lançar o \textit{benchmark} alvo, ficando depois à espera de um sinal para iniciar a medição de energia. Após configurar o ambiente necessário, o \textit{benchmark} envia um sinal de início ao orquestrador, espera 100 milissegundos e começa a execução da computação. Ao receber este sinal, o orquestrador ativa o PowerJoular para monitorizar o consumo energético do \textit{benchmark}, utilizando o seu identificador de processo, e permanece à espera do sinal de paragem. Quando o \textit{benchmark} conclui a computação, envia o sinal de paragem ao orquestrador, que termina a medição e armazena num ficheiro todas as informações relativas à execução. Este processo permite minimizar leituras desnecessárias e ruído, tendo em conta as limitações inerentes à medição de energia em aplicações.


Para ilustrar o funcionamento da \textit{framework} neste trabalho foram recolhidos dados para as coleções \texttt{List} e \texttt{Map} da linguagem Java, que serviram de base para o treino e validação dos modelos.
Após a obtenção dos perfis de energia, foi possível treinar modelos de aprendizagem automática supervisionada com os dados recolhidos. Nesta fase, foram comparados diversos algoritmos, como o \textit{Decision Tree Regressor}, \textit{Random Forest}, \textit{Gradient Boosting}, \textit{Linear Regression} e \textit{PySR}, tendo-se optado pelo PySR. Esta ferramenta gera um conjunto de expressões matemáticas ordenadas da mais simples para a mais complexa. Utilizando esta abordagem é possível obter equilíbrio entre desempenho e interpretabilidade, oferecendo modelos transparentes, ao contrário dos modelos de caixa preta, e facilitando a visualização da expressão utilizada para prever o consumo energético.
Por fim foi construida uma extensão para o VSCode que integra os modelos previamente treinados. Esta ferramenta combina análise estática com aprendizagem automática para realizar previsões de energia de programas. Esta combinação permite que o cálculo da energia utilizada seja bastante rápido, evitando a necessidade de executar programas, e aumentando a consciencialização dos programadores para a eficiência energética dos seus programas. A extensão apresenta uma interface gráfica intuitiva composta por vários contentores, onde cada contentor corresponde a um método do programa. Dentro de cada um, o utilizador encontra \textit{sliders} interativos que permitem ajustar os valores dos principais parâmetros de entrada, tais como variáveis ou ciclos relevantes para o método em questão. Ao modificar estes valores, a extensão recalcula e apresenta a previsão do consumo energético associada. Além disso, a mostra o valor total de energia estimado para todo o programa, oferecendo uma visão global do impacto energético do código em análise.

Os resultados obtidos mostram que a \textit{framework} consegue gerar modelos para programas personalizados necessitando apenas de alterações mínimas. A ferramenta de previsão de energia revelou-se útil para identificar variações relativas no consumo energético, à medida que os valores de entrada aumentam, observa-se um aumento correspondente nos valores de energia. No entanto, ao nível de valores absolutos, a precisão das previsões depende diretamente do desempenho dos modelos treinados, sendo necessária uma melhoria nestes para obter resultados mais precisos.

Em resumo, esta tese contribui para o campo da programação com consciência energética, combinando aprendizagem automática e análise estática para permitir uma previsão energética precisa e acessível. No entanto, a investigação e o desenvolvimento contínuos são essenciais para melhorar ainda mais a precisão do modelo, expandir a cobertura de funcionalidades e apoiar uma variedade mais ampla de cenários de programação.


\vfill

\begin{flushleft}
\textbf{Keywords:}
Computação ecológica, Análise estática de código, Previsão de energia, Software sustentável, Programação com consciência energética
\end{flushleft}

%\LIMPA
%\selectlanguage{portuguese}