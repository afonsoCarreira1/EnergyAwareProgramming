\chapter{Background}

To fully understand the processes involved in this work, it is important to first understand what energy profiling is, as it will be frequently referenced and used later.

Energy profiling is the systematic process of measuring, monitoring, and analyzing the power consumption of a system, using specialized software or hardware tools to collect detailed power consumption data. This data can be collected from different parts of a system, including applications, processes, and specific snippets of code, to provide insight into how different components and activities contribute to overall energy consumption. Through this type of power consumption profiling, developers are able to identify inefficiencies and optimize software to improve energy efficiency. This makes it very important for extending battery life in mobile devices, reducing operating costs in data centers, and also minimizing environmental impact through energy consumption. Energy profiling is a step toward making informed decisions to develop more sustainable and cost-effective computing solutions.

Energy profiling bridges the gap between traditional performance metrics and energy consumption, offering developers critical insights into how their software impacts system efficiency.
When programming, developers usually consider the time it takes to complete a program, or the response time from client to server, or the amount of memory it uses. Most don't have an idea of how much energy their program consumes, or how much it can consume in certain cases, and getting this idea is not as trivial as it seems. To get this awareness, measurements could be made, but they are also difficult to get compared to measuring the time a program takes, which is checking the differences in timestamps, or understanding the memory usage. As for measuring energy, the same program can produce different values due to its non-deterministic nature, meaning that results are often expressed as a range of possible values. Also, reducing the execution time of a program does not guarantee that power consumption will follow\cite{10.1145/3136014.3136031}. To obtain the measurements it is necessary to use software based tools that can facilitate this process, sometimes at the cost of less accurate readings or hardware devices for accurate values.

To perform hardware-based measurements, a power monitor or power meter device\cite{hackenberg2013power,ge2009powerpack} must be used to obtain the precise values that a system uses when connected to the electrical grid. These devices measure the power drawn from the grid to the machine, but they usually measure the power consumed by the entire system rather than by specific pieces of hardware, which means they have low granularity. Achieving granular measurements, such as isolating power consumption for specific components like the CPU or RAM, requires a more complex setup to avoid reading unnecessary power consumption. This approach is not suitable for developers who only want to analyze the energy performance of their programs, which shows the difficulty of measuring energy consumption.

An alternative with minimal overhead is to use software-based tools. These tools are typically easier to use, but may not provide values as accurate as hardware-based measurements. However, they are more versatile and can be implemented or modified to meet the user's needs. 

\section{Energy Tools} \label{sec:background_energy}



Understanding and optimizing energy consumption in software requires specialized tools capable of measuring and analyzing power usage. These tools provide valuable insights into how programs consume energy during execution, enabling developers to identify inefficiencies and optimize performance.

Intel RAPL (Running Average Power Limit)\cite{intel_rapl} is a tool for monitoring power consumption. It utilizes Model-Specific Registers (MSRs), which are used for program execution tracing, performance monitoring, and toggling CPU features. These registers can store the total energy usage of the CPU and memory, allowing it to be read and analyzed. Most software based tools rely on RAPL to measure energy consumption, since it is pretty accurate and is widely available in most CPUs.

PowerJoular \cite{noureddine-ie-2022} is an open source tool, capable of measuring energy, from the CPU and GPU, using the Intel RAPL power data through the Linux powercap interface it can read the energy from the CPU and for the GPU it uses NVIDIA SMI to directly read the power consumption.
To read the power consumption of specific processes, PowerJoular monitors the CPU cycles and utilization of each process. By knowing the total power consumption of the CPU through the RAPL interface, it can calculate the power usage of individual processes based on their CPU utilization.
It is build in ADA, that is considered one of most energy efficient programming languages\cite{PEREIRA2021102609}, and it can monitor applications by name or PID. In this work, it will be necessary to measure the energy consumption of programs, methods, and specific code snippets. To achieve this, PowerJoular will be employed as the primary tool for energy profiling.

Experiment-Runner\cite{S2_Group_Experiment_Runner} is a framework built in python made to facilitate experiments, it is easily customizable and can be used with energy measurement tools, such as PowerJoular, to monitor the power consumption of another process. While it offers robust support for energy-related experiments, some issues were identified during its use, which will be discussed in detail in Section~\ref{sec:preliminaryresults}.


\section{Code static and dynamic analysis} \label{sec:background_static_dynamic_analysis}

Static analysis, as the name implies, analyzes the code statically, meaning it examines the code without executing it. By examining the code, static analysis tools can understand how the program will behave at runtime\cite{ernst2003static}, this analysis often aims for soundness, meaning that if the tool catches an error, it means that the error really exists, there are no false negatives. However, this can come at the cost of producing false positives, where issues that are not actually problems are flagged, so it's important to keeps a balance between them. This analysis allows to check the entire source code and every path, much like compilers check syntax and types. Still, they can only predict some behaviors, as some can only be found when the program is executed, for example, by using dynamic analysis.

Dynamic analysis, on the other hand, executes the program and observes its exact behavior without having to estimate or predict. This type of analysis leaves no doubt about memory usage, output, the path taken, how much time it took\cite{ernst2003static}. A good example of dynamic analysis is unit testing, which tries to cover as many code paths as possible with different inputs, to understand as much as possible how the program works, and to find something that might be difficult to find with static analysis. However, dynamic analysis can be time-consuming, especially for programs that take a long time to complete.

For fast power estimation, static analysis is preferable. It analyzes code faster and is better suited for large projects with multiple dependencies, where dynamic analysis can be very difficult to achieve due to complex setup and long execution times. Although static analysis may not be as accurate as dynamic analysis, it is still a viable solution. In addition, static analysis is more portable because its setup is much simpler than the more complex setup required for dynamic analysis.
Developers may not have the time or the infrastructure to run the program just to get an average measure of energy consumption for a code snippet or program. Therefore, using static analysis to infer energy consumption makes sense in this context.

The use of static analysis implies the use of a parsing technique. This technique involves analyzing the syntactic structure of the provided code, respecting the rules of the language in which it is written. First, it is necessary to perform a lexical analysis to obtain the keywords, identifiers and tokens that the language contains. Then the parser uses these tokens together with the grammar rules of the programming language and outputs a tree. The output is an Abstract Syntax Tree (AST), which contains the logical structure of the code and allows further analysis. In the context of this work, this technique allows analyzing, for example, Java code and obtain its structure to find out which statements have been used.


The tool proposed in this work will primarily rely on static analysis to achieve its objectives. Static analysis, which examines code without executing it, is particularly effective in providing early insights into potential energy inefficiencies during the development process. By evaluating all possible code paths and scenarios, it avoids the dependence on specific runtime conditions inherent in dynamic analysis, making it a powerful tool for identifying and addressing energy-related issues without the complexity of real-time monitoring.

However, to create the energy consumption dataset required to train the ML models, this work will utilize dynamic analysis. Dynamic analysis involves executing the code under various conditions to gather real-world energy usage data. This approach enables the collection of accurate and context-aware energy consumption measurements, which are crucial for building a reliable dataset to inform and enhance the energy optimization models. By combining the strengths of both static and dynamic analysis, this work aims to develop a comprehensive framework for energy-efficient software design.


\section{Machine Learning} \label{sec:background_machine_learning}

Using a machine learning (ML) model to estimate energy consumption can offer advantages over a traditional approach. While traditional approaches such as empirical estimates based on historical data may work, they may not be the best solution in this case due to the unpredictable behavior of energy. Using an ML algorithm can help identify more complex patterns and provide a highly accurate estimate while adapting to the arrival of new information.

To build the energy inference function, a ML model will be used. It's important to understand how different ML algorithms work and which ones are best suited for the proposed project. There are several ML algorithms, and they fall into four main categories\cite{sarker2021machine}: supervised learning, unsupervised learning, semi-supervised learning and reinforcement learning.

The supervised ML requires labeled data for the model to train on. During training, the model has access to the input and output parameters, and it will try to match inputs to the correct outputs. It has 2 categories: classification, where it predicts discrete labels, such as whether a picture is a cat or a dog; and regression, where it predicts continuous values, for example, predicting the price of a house based on location, size, and so on or predicting energy consumption. These models can be very accurate, but they can also make incorrect predictions for patterns they were not trained on.

In unsupervised ML, the model attempts to find patterns and relationships in the unlabeled data set. With this technique, it is possible to find similarities or clusters in the data, for example, to detect an anomaly in the data. This technique does not require the effort of acquiring labeled data, but also it will be harder to understand if the output is correct or not.

Semi-supervised learning, as the name implies, uses a combination of supervised and unsupervised learning techniques. This hybrid approach is very useful in real-world scenarios where parts of the data can be labeled and others can't, allowing for better predictions in the output.

Reinforcement learning, where the model receives different feedback for different tasks and uses this feedback to perform the tasks in the most optimal way. The feedback can be in the form of rewards or penalties so that the model can better understand whether it is doing the task correctly. For example, a model playing a video game is rewarded for completing levels faster and penalized for failing the level. This method of learning allows for complex solutions to sequence-based problems, such as robotics or gaming. However, it can be time-consuming and computationally expensive.

There are some algorithms that can meet the proposed model requirements, such as: Linear regression, Tree-Based Models (Random Forest, Gradient Boosting), Neural Networks, Gaussian Processes, Support Vector Machines (SVM) and Genetic Programming.

The most common approach in ML is to use a linear regression algorithm, which is the simpler to implement and very effective. It can predict a continuous output based on the input independent variables. Linear regression is computationally efficient, easy to implement, and works well when the relationship between the input features and the output is linear. However, its simplicity is also its limitation, as it struggles with nonlinear relationships and can underperform when the input features interact in complex ways. It is also sensitive to outliers, which can significantly skew the results.

Tree-based models rely on decision tree models, which are used for structuring decisions, where in each branch a decision is made based on some criteria, and the end of the branch contains the final output. Random forest is a tree-based model that combines multiple decision trees to build an accurate model. When a prediction is needed, all the decision trees provide a vote, in classification or an average in regression and the random forest combines them to give the final prediction. Each tree is trained in different subsets of the data. This algorithm provides high accuracy, robustness to overfitting and estimates the features' importance, however, it has a higher computational cost, more memory usage, and it can take more time to reach a prediction than other approaches.

Gradient boosting is also a tree-based model, it builds trees (weak learners) sequentially with each of them trying to correct the errors of the previous one. It starts with a simple model and iteratively adds trees to reduce residual errors from the previous trees. It is generally more accurate than a random forest, however it is more prone to overfitting if there is a lot of noise in the data.

Neural networks models are particularly good at capturing complex, nonlinear relationships between inputs and outputs. A neural network consists of an input layer, hidden layers, and an output layer. The input layer receives the features (e.g., program attributes derived from the AST), the hidden layers process these features using weights, biases, and activation functions, and the output layer provides the final prediction, such as energy consumption. Neural networks are highly flexible and can adapt to various problem domains, automatically learning feature representations without requiring extensive manual engineering. However, they require large datasets to avoid overfitting and significant computational resources for training.

Gaussian Processes are particularly interesting, because they take into account the probabilistic nature of energy measurement and provide a range of possible values alongside with the probabilities. This makes them ideal for tasks where understanding the uncertainty in predictions is important, such as energy modeling. However, their computational complexity grows significantly with the size of the dataset, making them less practical for large-scale problems.

Support Vector Machines (SVMs) are a powerful tool in machine learning, capable of performing both classification and regression tasks. This algorithm identifies the optimal hyperplane in an N-dimensional space where it can separate all the features. When it is difficult to separate the features, a technique called kernel trick can be used to create an additional dimension to help separate them. This makes them good for high dimensional spaces, the ability to handle nonlinear relationships and the ability to ignore outliers. However, it can be harder to train this model, and tune the parameters.

Another alternative is to use genetic programming, which is not exactly a conventional ML algorithm, but rather a technique that can be applied to solve ML problems. It is a form of artificial intelligence inspired by the process of natural selection and evolution, where potential solutions to a problem are represented as programs or symbolic expressions. These programs improve iteratively, over generations, through mutations and crossovers, that sometimes can be random, guided by a fitness function. Genetic programming is useful for tasks like symbolic regression and feature classification. However, it can be computationally expensive and can produce inconsistent results due to its uncertain nature.

These algorithms will be taken into account when developing the energy inference function.