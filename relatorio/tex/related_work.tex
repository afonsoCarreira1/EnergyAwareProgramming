
\chapter{Related Work}\label{chapter:related_work}

This chapter discusses relevant approaches to energy measurement in software, its importance, and how to achieve it effectively. The chapter is divided into three categories: general context and approaches, static analysis-based tools, and dynamic analysis-based tools. This structure allows for a clearer comparison of technical strategies, their strengths, and their limitations.

\section{General Context and Approaches}

Energy efficiency is a critical focus across industries, as it directly impacts global sustainability, economic costs, and product quality.  The goal is to reduce greenhouse gases to create a sustainable future, reduce infrastructure costs, and improve product quality~\cite{annurev:/content/journals/10.1146/annurev.resource.102308.124234}. 

In particular, large scale computation and communication consume a lot of global energy, and these values have been increasing in the last decades, so the topic of energy aware programming and energy efficient software has been targeted by many researchers in recent years with the objective of reducing energy costs in large IT infrastructure~\cite{8880037}.
This improvement can be considered an optimization problem and can be tackled in several ways for example a heuristic approach by adjusting the hardware performance dynamically, or completing tasks in their deadlines, using the least energy possible. However, some of these implementations can only be short term solutions and in long term, the focus will be toward more complex models that can predict and optimize performance relative to hardware configurations~\cite{10.1145/1666420.1666438}. 

An approach to increasing developer's awareness of the energy consumption of their code involves creating extensions to already used programming languages, such as Java. For example, ECO~\cite{7194624}, a programming model as a minimal extension of Java. By rewriting some parts of the code to this extension syntax it is possible to define resource limits on the battery or temperature implementing adaptive behaviors through modes, and leveraging runtime monitoring.

In addition, new languages can be developed to address these goals, as demonstrated by ENT~\cite{10.1145/3062341.3062356}. ENT is a Java extension that empowers programmers with more direct control over the energy consumption of their applications. ENT’s type system enables applications to adapt dynamically to power constraints by switching operational modes based on resource availability, such as battery level or CPU temperature, allowing for software-level energy optimization. However, the language introduces complexity, making it potentially challenging for developers to learn and adapt to existing codebases. 

Using machine learning algorithm has also shown to be effective to estimate energy consumption. Fu et al.~\cite{8726531} used four distinct ML algorithms (Ridge Regression, Linear Regression, Lasso, and Random Forest) to analyze the energy consumption of various apps, achieving low average error rate. These findings demonstrate the potential of such models to serve as the foundation for future tools, enabling developers to predict and optimize software energy usage without relying on specialized hardware. Other works go even further with using machine learning for energy efficiency and explore LLMs for generating not only optimized code but even reviewing and refactoring software for sustainability. Rani et al.~\cite{rani2025can} demonstrates that while LLMs recommend a range of optimizations, from vectorizing to memory management, their effectiveness in terms of actual energy savings remains low with LLMs usually making trade-offs between memory and execution time. Cappendijk et al.~\cite{cappendijk2024generating} investigate prompt-based optimizations by modifying LLM inputs with sustainability-focused instructions. While some prompt–model–task combinations achieved up to 59\% reduction in energy consumption, the results were highly inconsistent, with some prompts even increasing energy usage by over 400\%. Their findings underline both the potential and current limitations of using LLMs to directly influence software energy efficiency through prompt engineering. 


Estrada et al.~\cite{estrada2022learning} proposed an energy consumption prediction model to optimize energy management in cloud and fog infrastructures, addressing challenges such as high operational costs and environmental impact. Their system integrates machine learning with sensor-based hardware to create a non-intrusive monitoring approach. Using a network of sensors to collect real-time data on metrics like voltage and power, processed via MQTT and visualized on dashboards, the study employed a robust linear regression model to predict hourly energy consumption. The research emphasizes the importance of real-time monitoring and machine learning integration for achieving energy efficiency in data centers, aligning with Green IT principles.

\section{Static Analysis Tools}

The use of static analysis can be valuable for understanding how instructions affect the energy consumption of programs. Aggarwal et al.~\cite{aggarwal2014power} shows that system calls are directly related to energy consumption in Android applications. With this insight, it's possible to use static analysis to identify system calls within the code. This information can then be used to infer potential energy usage patterns, providing an early indication of where higher energy consumption may occur. This approach highlights the importance static analysis can have to understand program energy behaviors.

To tackle the problem of energy consumption in IT, some solutions have been presented. Some researchers focused on using energy measurement tools, like JRAPL to measure common libraries in Java and understand how much energy they use and what are the best alternatives to improve the energy efficiency of the code~\cite{10.1145/2896967.2896968}. Observing common libraries for the implementation of list, sets and maps, is possible to see which ones have the better energy efficiency and what changes could improve the code.
Hasan et al.'s~\cite{10.1145/2884781.2884869} research adds to this by creating detailed energy profiles for various Java collection classes, including lists, maps, and sets, across different implementations (Java Collections Framework, Apache Commons Collections, and Trove). Their work presents concrete quantification of energy consumption in these collections based on common operations such as insertion, iteration, and random access, and highlights the performance impact of collection types on energy efficiency for different input sizes.

However, because these collections are often used with threads, it is important to understand how much energy efficiency can be improved without compromising thread safety. The energy consumption of Java's thread-safe collections was studied by Pinto et al.~\cite{7816451}, where researchers demonstrated that switching to more energy-efficient collection implementations can reduce energy usage while maintaining thread safety.

Building on these efforts, Pereira et al.~\cite{10.1145/3238147.3240473} introduced a static analysis tool (Jstanley), as part of an Eclipse plugin, that can detect energy inefficient collections and recommend better alternatives. While Jstanley demonstrated notable improvements in energy efficiency within its specific context, it has several limitations.
For example, they only account for three collections interfaces, (Lists, Sets and Maps) and only account for three different sizes (25,000, 250,000 and 1,000,000), it does not account for loops, thread safe and thread unsafe collections. Compared to our approach, Jstanley is limited, as it does not provide the actual information about the energy spent, it just shows recommendations. While the tool shows great improvements in its tested environment, replacing collections may not be enough in many practical cases. A more extensive tool, capable of analyzing a wider range of collections and providing energy metrics, would enable developers to achieve even greater energy efficiency and awareness.

Oliveira et al.~\cite{8816747}, proposed a tool (CT+) that is capable of performing static analysis of the code, recommend changes that reduce energy consumption and automatically apply these changes. It is achieving up to 16.34\% reduction in energy consumption even for a real-world, mature system such as Tomcat, it could reduce the energy consumption about 4.12\%. It improved from previous works by taking into account more collections implementations, more operations, thread safety and support for mobile applications. Although CT+ is a significant improvement over previous tools, it still has some limitations. For example, it does not provide the actual energy consumption of the code, it just shows recommendations based on the previously collected statistics, as it did not leverage machine learning. Additionally, it only works with collections, and it is not easily generalizable for other context. 

\section{Dynamic Monitoring and Profiling Tools}

In addition, SEEP~\cite{10.1145/2094091.2094106} uses symbolic execution for energy profiling, generating multiple binaries representing different code paths and input scenarios. By analyzing these binaries with hardware-based energy measurement devices, SEEP provides energy consumption data, offering a deeper understanding of code efficiency across various inputs and paths. This approach complements other tool, called PEEK~\cite{187026}, which builds on SEEP to help developers optimize energy usage with minimal effort. PEEK is an IDE-integrated framework that guides developers in writing energy-efficient code. It has a front end for IDE interfaces (e.g., Eclipse, Xcode), a middle end to manage data and versioning via Git, and a backend where energy analysis is performed, either through SEEP or hardware devices. Through these layers, PEEK identifies inefficiencies and suggests optimizations, supporting efficient coding practices. However, it has some limitations when compared to the proposed approach in this work, it uses dynamic analysis instead of static analysis, which was already explained in Section~\ref{sec:background_static_dynamic_analysis}, why it was chosen over dynamic analysis. Additionally, it does not incorporate any machine learning techniques.

Some command tools, that work on Linux, help facilitate the process of energy measurement, like Perf~\cite{perfwiki_main}, that is a command line tool already available in Linux, and is mainly used for performance monitoring and profiling. Although it's not specific for energy measurement, it can do it, with Intel RAPL but not as practical as other tools, specially when it's needed to measure a single process energy consumption. Powertop~\cite{archlinux_powertop} is another tool capable of providing the power consumption, however it only works for laptops, as it requires to check the battery to see how much energy was used and calculate the power consumption.

JoularJx~\cite{noureddine-ie-2022} is a Java agent that attaches to the Java Virtual Machine (JVM) at startup to monitor energy consumption. It runs in a separate thread, collecting CPU usage data for the JVM, its threads, and individual methods using statistical sampling. JoularJx performs power estimation by using platform-specific tools, such as the Intel API through the Linux RAPL interface, or a custom regression model for Raspberry Pi devices. It periodically analyzes stack traces to isolate energy consumption at the method level, taking into account execution paths and separating application-specific calls from system or agent-induced calls.

\section{Limitations of Existing Approaches}

The reviewed solutions showed strong potential in energy-aware software development, showcasing effective strategies such as including programming languages extensions, machine learning models, the use of static and dynamic analysis. However, these solutions have some limitations, like the need to execute the code before showing the average energy cost to the developer, having limited collections or lacking integration with the development workflow. These limitations point to the need for a more accessible, less limited solution.

The objective of this work aimed to develop a process that could obtain models that statically estimate program energy consumption, by combining established best practices with unique, domain-specific techniques. The final models produced by the framework pipeline can be trained using various language-specific collection libraries, and they are flexible enough to be modified or enhanced as needed. The final extension tool will then display the energy consumption of the code being analyzed making the developer more aware of the program's energy usage and helping him make better decisions.
