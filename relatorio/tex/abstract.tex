% ----------------------------------------------------------------------
% Página do resumo em Inglês:
\selectlanguage{english}
\vspace*{2cm}
\begin{center}
\Large \bf Abstract
\end{center}
\vspace*{1cm} \setlength{\baselineskip}{0.6cm}

  \wo{update this}
  Energy efficiency in software development is on the rise due to the increasing demand for environmental sustainability and the need to reduce technology operating costs. Despite progress, developers often lack the tools and knowledge to effectively optimize the energy consumption of their programs. 
  This project introduces a framework for automating the construction of energy prediction models for Java methods, standard library code and user code. The pipeline begins with the automatic generation of Java programs, each designed to exercise specific methods. The programs are executed and dynamically profiled using the PowerJoular tool for estimating energy consumption. Profiling-based features extracted on the run are exploited for training machine learning models capable of predicting energy consumption given static code attributes. The resulting models are integrated into a VSCode extension that statically estimates code snippets real-time energy consumption. The tool aims to raise awareness and support energy-aware software development practices through immediate feedback on energy consumption.

%Resumo até \textbf{300} palavras. 

\vfill

\begin{flushleft}
\textbf{Keywords:}
Green computing, Static code analysis, Energy prediction, Sustainable software, Energy-aware programming
\end{flushleft}

%\LIMPA
\selectlanguage{portuguese}
% Fim da página do resumo em Inglês.
% ----------------------------------------------------------------------