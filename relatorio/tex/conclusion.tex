% Conclusao

\chapter{Conclusion}\label{chapter:conclusion}


The use and management of energy has become a global issue, and the energy consumed by information and communications technology has a significant impact at a global scale. When developing programs, developers can have difficulties on how to improve the energy efficiency of their programs, due to the complexity and limited accessibility of relevant information as well as the lack of supporting tools and guidelines. 
This thesis reviews and builds upon prior research, tools, and techniques, by proposing an improved framework for estimating energy consumption. The framework integrates machine learning with static analysis to generate predictive models capable of estimating the energy usage of software programs.
The framework presented is divided in four modules, and it eases the process of obtaining the estimation models, by generating the necessary data, collecting the energy profiles, training the dataset and presenting the results. By simply providing a program or selecting a Java collection, the framework can generate estimation models adapted to the input. 

For developers who prefer not to wait for the full analysis process or are less familiar with energy-aware programming, our VSCode extension offers a quick and accessible alternative. By using the pre-trained models the tool provides immediate feedback on the estimated energy consumption of their code, without the need to execute it, helping raise awareness of energy efficiency in software development. 

During the experiments, the framework proved it was capable of evaluating custom programs, with minimal modifications, which makes it reliable to obtain more data from across a wider variety of codebases. Additionally, the extension tool also proved it can raise energy awareness by using the trained models and static analysis, to provide real-time feedback on the energy impact of code changes.

In summary, this thesis contributes to the field of energy-aware programming by combining machine learning and static analysis to enable accurate and accessible energy prediction. Nonetheless, continued research and development are essential to further improve model accuracy, expand feature coverage, and support a broader range of programming scenarios.
